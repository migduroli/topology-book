\chapter{Topological spaces}\label{ch:topological-spaces}

In this first chapter, we are going to study a fundamental concept which will pervade
the rest of this course, the concept of \emph{topological space}.
Once this concept is introduced, we will proceed with some basic topological concepts,
besides studying the most important operations which allow for the construction of new
topological spaces.

\section{Topologies, bases and subbases}
\label{sec:topologies-bases-and-subbases}

\subsection{Topologies}
\label{subsec:topologies}

The modern definition of topological space is based upon the fundamental concept of
open  set (which is axiomatically introduced).
Thus, a topological space is defined by specifying the family of open sets which form it.
The formal definition is as follows.

\begin{definition}
	\label{def:topology}
	Let $X$ be a set, and let $\T$ be a family of subsets of $X$ (i.e., $\T\subset\P X$).
	Then $\T$ is called a \textbf{topology} on $X$ if the following properties are
	fulfilled:
	\begin{enumerate}
		\item $\emptyset, X\in \T$
		\item Let $\{A_i\}_{i\in I}$ be a family of subsets such that $A_i \in \T$,
		then $\bigcup_{i\in I} A_i\in\T$
		\item Let $A, B\in \T$, then $(A\cap B)\in \T$
	\end{enumerate}
\end{definition}

Having the formal definition for \emph{topology}, it readily follows that a
\textbf{topological space} is a tuple $(X, \T)$, with $X$ being a set and $\T$ a
topology on $X$.
The elements of $\T$ are called the \textbf{open subsets} of $X$.

\begin{remark}
	The property (\defref{def:topology}-2) could be rewritten in a more concise way by
	saying that, if $\F\subset\T$, then $\bigcup\F\in\T$.
	Also, the third property can be readily extended to any finite amount of open sets
	by induction, so that $\forall \{A_k\}_{k=1,\dots,n}$ with $A_k\in\T$, then
	$\bigcap_{k=1}^n A_k\in \T$.
\end{remark}

For practical purposes, we might simply use $X$ instead of $(X, \T)$, keeping in mind
that $\T$ changes with $X$.
The first topological concept we introduce based upon the concept of \emph{open subset} is
the concept of \emph{neighbourhood}.

\begin{definition}
	\label{def:neighbourhood}
	Let $C, U\subset X$, with $(X, \T)$ being a topological space.
	Then, $U$ is said to
	be a \textbf{neighbourhood} of $C$ if $\exists A\in\T$ such that $C\subset A \subset U$.
	The subset $U$ is said to be a neighbourhood of a point $x\in X$ if $U$ is a
	neighbourhood of $\{x\}$.
	An open neighbourhood of $x$ is simply an open subset of $X$ containing $x$.
\end{definition}

The neighbourhoods of the points of a topological space determine the open sets of the topology:

\begin{theorem}
	\label{th:open-subset-neighbourhood}
	Let $(X, \T)$ be a topological space.
	A subset $U\subset X$ is an open subset \emph{if and only if} (iff) $U$ is neighbourhood of all its points.
\end{theorem}

\begin{proof} To prove this theorem will proceed with the demonstration of the left and
right implies.

\noindent{$\boxed{\Rightarrow}$} This is trivial, given that $U\in\T$ exists and
fulfils the conditions to be a neighbourhood of all its points.

\noindent{$\boxed{\Leftarrow}$} Let $U$ be a neighbourhood of all its points.
Then, $\forall x\in U$, $\exists V\in \T\,:\,\{x\}\subset V\subset U$ $\Rightarrow$
$U = \bigcup\F$ with $\F=\{V\in\T\,:\,V\subset U\}\,\Then\,U$ is open, since the union
of open subsets is open according to \defref{def:topology}-2.
\end{proof}

The intuitive idea we must have developed so far is that a neighbourhood of a point $x$ is a set
$U$ which contains all the points ``\emph{around}'' $x$, which also means that $U$ is open
\iff\,it only contains all the points surrounding every one of its points.

\begin{example}
	Let $X$ be a set, the \emph{trivial topology} is $\T=\{\emptyset, X\}$, and the
	topology $\T=\P X$ is called the $\emph{discrete topology}$.
\end{example}

The proof that both examples fulfil the conditions required by the definition
\defref{def:topology} is straightforward, and left to the reader if interested.
On the one hand, the trivial topology is the smallest one possible, since it has the
minimal elements required to fulfil the definition \defref{def:topology}.
On the other hand, the discrete topology is the largest one possible, given that all
the possible subsets of $X$ are open.

\begin{definition}
	The elements $x\in X$ will be called \emph{points} (or \emph{singletons}).
	Thus, a point $x\in X$ is said to be \emph{isolated} if $\{x\}$ is an open subset
	of $X$, i.e. $\{x\}\in \T$.
\end{definition}

\begin{remark}
	A trivial result is that a topological space $(X,\T)$ is \emph{discrete} (i.e.,
	$\T=\P X$) \iff\ $\forall x\in X$, $x$ is isolated.
\end{remark}

\subsection{Bases}\label{subsec:bases}

\begin{definition}
	\label{def:base}
	Let $(X, \T)$ be a topological space, and $\B = \{B \subset X\,:\, B \in \T\}\subset\T$
	a family of open sets.
	The family $\B$ is a \textbf{base} for $X$ \iff\, every open subset of $X$ can be
	expressed as the union of elements of $\B$.
	This is equivalent to say that, $\B\subset\T$ is a base for $(X,\T)$ if for every $U\in\T$ and $u\in U$, there is a $B\in \B$ such that
	$u\in B\subset U$.
\end{definition}

The convenience of introducing this concept becomes clear with the following two theorems:

\begin{theorem}
	\label{th:base-properties}
	Let $(X, \T)$ be a topological space, and $\B$ a base for $(X, \T)$.
	The following two properties are satisfied:
	\begin{enumerate}
		\item $\bigcup_{B\in\B} B \equiv \bigcup\B = X$
		\item Let $A, B\in \B$ and $x\in A\cap B$, then $\exists C\in \B$ such
		that $x\in C\subset (A\cap B)$
	\end{enumerate}
\end{theorem}

\begin{proof}
	This proof is straightforward by using \defref{def:topology} and \defref{def:base}:
	\begin{enumerate}
		\item \noindent{$\boxed{\subseteq}$}  $\bigcup\B \subset X$ is trivial.

		\noindent{$\boxed{\supseteq}$} Given that $X\in \T$ (hence it is an open
		subset) $\Then$ $X$ can be expressed as the union of a family $\F\subset \B$
		$\Then$ $X \subset \bigcup \F\subset \bigcup \B$.

		\item Let $A, B\in \B$ $\overset{\text{\defref{def:topology}}}{\Then}$
		$(A\cap B)\in \T$ $\overset{\text{\defref{def:base}}}{\Then}$
		$\exists C\in \B$ such that $x\in C\subset(A\cap B)$.
	\end{enumerate}
	With this, the theorem is proved to hold true.
\end{proof}

Interestingly, the two properties of the previous theorem are sufficient for a set
family to be a base for a topology:

\begin{theorem}
	Let $X$ be a set and $\B\subset \P X$ a family of subsets of $X$ such that:
	\begin{enumerate}
		\item $\bigcup \B = X$
		\item If $A, B\in \B$ and $x\in (A\cap B)$, there exists $C\in \B$ such that
		$x\in C\subset (A\cap B)$
	\end{enumerate}
	Therefore, there exists a unique topology $\T$ on $X$ with $\B$ being a base for it,
	and such topology is the smallest (w.r.t.\footnote{with respect to} the inclusion)
	for which the elements of $\B$ are open.
	\label{th:1-3}
\end{theorem}

\begin{proof}
	To proceed with the proof, we are going to first prove that there exists a topology $\T$, and later
	we will prove that $\T$ is unique.\\
	Let us define $\T = \{\bigcup \F\,:\,\F\subset \B\}$, and now proceed with the proof
	that $\T$ is a topology.
	\begin{enumerate}
		\item By condition (1), $X=\bigcup \B$ $\Then$ $X\in \T$.
		And, it is trivial that $\emptyset = \bigcup \emptyset \in \T$.

		\item Let $\F\subset \T$, and let $\F_A=\{B\in \B\,:\, B\subset A\}$ for each
		$A\in\F$, so that $A=\bigcup\F_A$.
		Let $\F^*=\bigcup_{A\in\F}\F_A\subset\B$.
		Then,  $\bigcup \F =\bigcup\F^*\in\T$.

		\item Let $A, B\in \T$, and $\F = \{C\in\B\,:\, C \subset(A\cap B)\}$.
		Let us prove that $A\cap B = \bigcup\F$:
		\begin{itemize}
			\item[\boxed{\supseteq}] This is a trivial proof: $\forall x\in\bigcup\F$ $\Then$
			$x\in\setdef{C\in\B}{C\subset(A\cap B)}\ \Then\ x\in (A\cap B)$ $\Then$
			$\bigcup\F \subset (A\cap B)$.
			\item[\boxed{\subseteq}] Let $x\in (A\cap B)$, according to $(\emph{2.})$,
			$\exists C\in \B$ such that $x\in C\subset (A\cap B)$ $\Then$ $x\in C\subset
			\bigcup\F$ $\Then$ $(A\cap B)\subset \bigcup\F$.
		\end{itemize}
		Finally, $\bigcup \F\in \T$ $\Then$ $(A\cap B) \in \T$.
	\end{enumerate}
	Therefore, $\T$ is a topology on $X$.
	To prove the final part of the theorem, consider $\T'$ a topology on $X$ with
	$\B$ as base.
	By definition, $\T\subset \T'$.
	Therefore, $\T$ is unique and minimal.
\end{proof}

Thus, to define a topology on a set $X$ one only needs to determine a family $\B$ with
the properties just described.
In what follows, we introduce a related concept which
will be really convenient later on this book, the one of \emph{neighbourhoods base} or
\emph{neighbourhoods system}.

\begin{definition}
	\label{def:neighbourhoods-base}
	Let $(X, \T)$ be a topological space, and $x\in X$.
	An \textbf{(open) neighbourhoods base} for $x$ is a family $\B\subset\T$ of open
	neighbourhoods of $x$ such that, for every $U$ neighbourhood of $x$, $\exists B\in\B$ such that
	$x\in B\subset U$.
\end{definition}

It is clear that if $(X,\T)$ is a topological space, and we have a family
$\{\B_x\}_{x\in X}$ so that $\B_x$ is a neighbourhoods base for $x$, then
$\B=\bigcup_{x\in X}\B_x$ is a base for (X, $\T$).
Reciprocally, let $\B$ be a base for $(X, \T)$, then $\B_x = \setdef{A\in\B}{x\in A}$ is
a neighbourhoods base for $x$.

\begin{remark}
	In some references, the concept of \emph{neighbourhoods base} is also called
	\emph{neighbourhoods system}, and it is also usual to make a difference between
	open neighbourhoods bases and neighbourhoods bases, the former usually called
	\emph{local bases}. In this text we will not make such distinction, and we will
	always refer to both as \emph{neighbourhoods base}.
\end{remark}

In this context, it is useful to introduce the concept of \emph{filter},
which has as special case the concept of \emph{neighbourhoods base}:

\begin{definition}
	\label{def:filter}
	Let $(X, \T)$ be a topological space, and $\B$ a family of subsets of $X$. $\B$
	is a \textbf{filter} for $X$ \iff\ the following properties are fulfilled:
	\begin{itemize}
		\item $X\in\B$ and $\emptyset\notin\B$
		\item If $A, B\in\B$, then $A\cap B\in\B$
		\item If $A,B\subset X$, $A\in\B$ and $A\subset B$, then $B\in\B$
	\end{itemize}
\end{definition}

It is easy to see that a neighbourhoods base $\B$ for $x\in X$ is a filter, since
$\B$ is a family of open sets, and it is closed under finite intersections.


\subsection{Subbases}\label{subsec:subbases}

In this section we will introduce a concept similar to \emph{base}, which will also be
useful when defining topologies.

\begin{definition}
\label{def:subbase}
	Let $(X, \T)$ be a topological space.
	A family $\S\subset \T$ is a \textbf{subbase}	for $(X, \T)$ \iff\ the family of all finite intersections
	of elements of $\S$, namely	$\I(\S)$, is a base for $(X, \T)$\footnote{The family of all finite intersections of
	elements of $\S$ means: $\I(\S) = \bigcap_{S\in\mathcal{F}} S$, for all finite subsets $\mathcal{F}$ of $\S$. In particular for all subsets $\mathcal{F}$ of cardinality 0 (empty set, $\mathcal{F}_0=\emptyset$) or 1 (singletons, $\mathcal{F}_1=\{A\}$).}.
\end{definition}

The difference between subbases and bases is that the family of subsets does not have
to satisfy any specific condition to be a subbase:

\begin{theorem}
	Let $X$ be a set, and $\S\subset \P X$ a family of subsets of $X$.
	There exists a unique topology on $X$ with $\S$ as subbase, and it is smallest
	(w.r.t.\,the inclusion) for which the elements of $\S$ are open.
	\label{th:1-4}
\end{theorem}

\begin{proof}
	To prove this result we will show that:
	\[
	\B\doteq \I(\S)=\left\{\bigcap_{S\in\mathcal{F}} S=\bigcap\mathcal{F}\,:\,\mathcal{F}\subset \S\right\},
	\]
	satisfies the conditions of \thref{th:1-3}, as follows:
	\begin{enumerate}
		\item $\mathcal{F}_{0}=\emptyset$ is a finite subset, and $X=\bigcap \emptyset\in\B$, therefore $X\subset
		\bigcup\B$.
		Besides, all other finite subsets are $\mathcal{F}_{1}=\{A\}$ with $A\in\S
		\subset\P X$, $\bigcap\mathcal{F}_{1}\subset X$.
		Therefore $\bigcup \B\subset X$.
		Thus $\bigcup\B = X$.
		\item Let $A, B\in \B$, and $x\in (A\cap B)$.
		By definition, $A\in\B \Rightarrow A=\bigcap S$,
		and $B\in \B \Rightarrow B =\bigcap T$, with $S, T\subset \S$.
		Then, $A\cap B = \left(\bigcap S\right)\cap\left(\bigcap T\right)=\bigcap (S\cup T)$.
		It is then clear that $\exists C\in\B$ such that $x\in C\subset (A\cap B)$.
	\end{enumerate}
	Thus, the conditions for \thref{th:1-3} are fulfilled which means that
	$\B$ is base for a topology in $X$, with $\S$ as a subbase.
	Finally, to prove the uniqueness, let us consider a topology $\T'$ on $X$, 
	such that $\forall S\in \S, S\in \T'$ $\Rightarrow$ $\forall B\in\B, B\in\T'$ 
	$\Rightarrow$ $\T\subset \T'$, which also proves that $\B$ is the smallest.
\end{proof}

\begin{example}
	\footnote{This will be discussed in further detail later on the book, when
	we equip any ordered space with a topological structure.}
	The \textbf{order topology} on $\mathbb{R}$ is the topology generated by the
	base $\B=\setdef{(a,b)}{a,b\in\mathbb{R}, a<b}$. At the same time, we also have
	the usual-metric induced topology on $\mathbb{R}$, which is the topology
	generated by the base $\B=\setdef{B_\epsilon(x)}{\epsilon>0, x\in\mathbb{R}} = \setdef{(x-\epsilon, x+\epsilon)}{\epsilon>0, x\in\mathbb{R}}$.
	It turns out that both the order topology and the metric topology are the same.
	This is a consequence of the fact that every open of the metric topology is an
	open interval, hence it is also an open set of the order topology. And, conversely,
	if $(a,b)$ is an open on the order topology, and $x\in\mathbb{R}$, then can simply
	take $\epsilon=\min\{x-a, b-x\}$, and we have that $B_\epsilon(x)=(x-\epsilon, x+\epsilon)\subset (a,b)$.
	Therefore, all the open of the order topology are open of the metric topology, and
	vice versa.
\end{example}

\subsection{Product topology}\label{subsec:product-topology}

At this point, it seems natural to try and generalise the concept of topological space to
the cartesian product of topological spaces. The idea is that we have a family of
topological spaces $\{(X_i, \T_i)\}_{i\in I}$ (sometimes we will refer to them simply
as $\{X_i\}_{i\in I}$), and we want to equip the cartesian product
$X=\prod_{i\in I} X_i$ with a topological structure.
Before we proceed with the formal introduction of any topology of the product space,
such as the na\"{i}ve \emph{box topology}, or
the \emph{product topology} (or,
\emph{Tychnoff topology} due to the fact that it was introduced by Andrey Tychnoff in 1935\footnote{%
	Tikhonov, Andrey Nikolayevich (1935). \emph{\"{U}ber einen Funktionraum}, Mathematische Annalen (111): 762--766.
	Cited in Hocking, John G. \& Young, Gail S. (1961). Topology. Addison-Wesley.%
}), we will briefly introduce the concept in the more familiar context of $\mathbb{R}^2$.

\subsubsection{Neighbourhoods in $\mathbb{R}^2$}\label{subsubsec:neighbourhoods-in-r2}
We expect the reader to be familiarised with the concept of the Euclidean (or, simply, \emph{natural}) distance
in $\mathbb{R}^2$, although this will be thoroughly discussed when we deal with metric spaces in Chapter~\ref{ch:metric-spaces}.

The Euclidean distance between two points $x=(x_1,x_2)$ and $y=(y_1,y_2)$ in $\mathbb{R}^2$ is defined as follows:
\begin{equation}
	d(x,y) = \sqrt{(x_1-y_1)^2+(x_2-y_2)^2}.\label{eq:euclidean-distance}
\end{equation}
which represents the length of the segment joining $x$ and $y$.
This distance can be used to define the \emph{open ball} centred at $x$ with radius $\epsilon$ as follows:
\begin{equation}
	B_\epsilon(x) = \{z\in\mathbb{R}^2\,|\,d(x,z)<\epsilon\}.
	\label{eq:open-ball}
\end{equation}
which certainly contains $x$, and all the points \emph{around} $x$ at a distance.
And, given that it contains all the points \emph{around} $x$, it is clear that $B_\epsilon(x)$ is an open set
in $\mathbb{R}^2$.

\begin{remark}
	Using the standard notation $(a,b)\subset\mathbb{R}$ to refer to the
	open interval between $a$ and $b$ would not be convenient, since we are using
	the same notation for coordinates in $\mathbb{R}^2$.
	Whenever this potential misunderstanding is possible, we will use the notation
	$I_{(a,b)} = \setdef{x\in\mathbb{R}}{a<x<b}$.
\end{remark}

However, if our main concern is to determine subsets of $\mathbb{R}^2$ which contain all the points
\emph{around} $x\in\mathbb{R}^2$, we can either use open balls or, instead, open rectangles centred at $x$:
\begin{align*}
R_\epsilon(x) =& \{z\in\mathbb{R}^2\,|\,|x_1-z_1|<\epsilon_1, |x_2-z_2|<\epsilon_2\} \\
=& I_{(x_1-\epsilon_1, x_1+\epsilon_1)}\times I_{(x_2-\epsilon_2, x_2+\epsilon_2)}
\end{align*}
given that $R_\epsilon(x)$ also contains all the points which are intuitively \emph{around} $x$, hence it is also an
open set in $\mathbb{R}^2$.

These open rectangles generalise to higher dimensions in the obvious way, where the term \emph{open box} seems more appropriate. The main advantage of using open boxes, instead of using open balls, is that they allow us to come up with an easy procedure to equip the product of two topological spaces $X \times Y$ with a topological structure. This procedure is based on the observation that the product of open sets $U \times V$ form a base for the product topology on $X \times Y$. And, indeed, this observation is true for any finite product of topological spaces. Unfortunately, this seemingly simple procedure does not work for infinite products of topological spaces, as was noticed by Tychnoff in 1935. The problem was that the product of open sets in an infinite product of topological spaces does not satisfy certain \emph{desirable} properties which we will discuss later. Tychonoff found a way to overcome this problem, which gave rise to the concept of \emph{product topology}.

\subsubsection{Product topology: Tychonoff's definition}\label{subsubsec:product-topology}

Before being able to introduce the definition for the product topology, we need to
introduce the concept of cartesian \emph{projection}.

\begin{definition}
	\label{def:projection}
	Let $\{X_i\}_{i\in I}$, with $I\subset\mathbb{N}$, be a family of sets, and
	$X=\prod_{i\in I} X_i$ their cartesian product.
	The \textbf{cartesian projection} (or, simply, projection) is the function defined as follows:
	$$
  	\begin{array}{rrcl}
		\pi_j : & X &\longrightarrow & X_j\\
		 & x &\longmapsto     & \pi_j(x)
  	\end{array}
	$$
	where $\pi_j(x)=x_j \in X_j$ is the $j$-th coordinate of $x$.
	This definition can be extended to any subset $A=\prod_{i\in I}A_i\subset X$
	by defining $\pi_j[A] = \setdef{\pi_j(x)}{x\in A}\linebreak = A_j\subset X_j$.
	The \emph{inverse image mapping} induced by the projection $\pi_j$ is defined as follows:
	$$
		\begin{array}{rrcl}
			\pi^{-1}_j : & X_j &\longrightarrow & X\\
			 & x_j &\longmapsto     & \pi^{-1}_j(x_j)
		\end{array}
	$$
	where $\pi^{-1}_j(x_j)=\{x\in X\,:\,\pi_j(x)=x_j\} \subset X$.
	This can also be expressed as $\pi^{-1}_j(x_j) = \{x_j\} \times \prod_{i\in I\setminus\{j\}}X_i \subset X$.
	This definition can also be extended to any subset $A_j\subset X_j$ by defining
	$\pi^{-1}_j[A_j]=\setdef{\pi^{-1}_j(x_j)}{x_j\in A_j}=A_j\times\prod_{i\in I\setminus\{j\}}X_i\subset X$.
\end{definition}

From the above definition, we can deduce the following lemma which will be useful later on to prove some results:

\begin{lemma}
	\label{lem:projection}
	Let $\{X_i\}_{i\in I}$ be a family of sets, and $X=\prod_{i\in I} X_i$ their cartesian product.
	Then, the following properties hold true:
	\begin{enumerate}
		\item Let $A_j\subset X_j$, then $(\pi_j\circ\pi^{-1}_j)[A_j] = \text{id}_{A_j}$.
		\item Let $A\subset X$, then $A = \bigcap_{j\in I}(\pi_j^{-1}\circ\pi_j)[A]$.
	\end{enumerate}
\end{lemma}
\begin{proof}These are very simple results, which only require the use of the definitions \defref{def:projection}:
	\begin{enumerate}
		\item Let us use the definition of the inverse first, we have $\pi_{j}^{-1}[A_j]=\{ x\in X\,:\,\pi_{j}(x)=x_j\in A_j\}$.
		Then, $(\pi_{j}\circ\pi_{j}^{-1})[A_j]=\{\pi_{j}(x)\,: x\in \pi_{j}^{-1}[A_j]\}=\{x_j\in A_j\} = A_j$.
		\item {$\boxed{\subseteq}$} Let $x\in A$, then $x\in \prod_{i\in I} A_i$. Given that $\pi_j^{-1}[A_j] = A_j\times\prod_{i\in I\setminus\{j\}}X_i$,
		we have that $x\in \pi_j^{-1}[A_j]$ for each $j\in I$. Therefore, $x\in \bigcap_{j\in I}(\pi_j^{-1}\circ\pi_j)[A]$.\\
		{$\boxed{\supseteq}$} Let $x \in \bigcap_{j\in I}(\pi_j^{-1}\circ\pi_j)[A]$, then $x\in \pi_j^{-1}[A_j]$ for each $j\in I$.
		This means that $x_j \in A_j$ for each $j\in I$, hence $x\in \prod_{i\in I} A_i$.
	\end{enumerate}
\end{proof}

\begin{definition}
	\label{def:product-topology}
	Let $\{(X_i, \T_i)\}_{i\in I}$ be a family of topological spaces, and
	let us consider the cartesian product $X=\prod_{i\in I} X_i$.
	The \textbf{product topology} on $X$ (i.e., $\T_{X}$) is the topology generated by the subbase
	defined as follows:
	$$
		\S = \setdef{\pi^{-1}_j[U]}{U\in\T_j, j\in I}.
	$$
	I.e., $\T_X$ is generated by the subbase $\S$, with the subbase elements
	being the inverse projections of open sets of $\T_j$.
\end{definition}

The following theorem is an important result, not only because it gives us the recipe
for the construction of a base for the product topology, but also because it shows that
the base open sets for the product topology are not the cartesian products of open sets
of the underlying topological spaces.

\begin{theorem}
	\label{th:product-topology-base}
	Let $\{(X_i, \T_i)\}_{i\in I}$ be a family of topological spaces, and let $\{\B_i\}_{i\in I}$ be
	a family such that $\B_i$ is a base for $\T_i$. Then, a base for the product topology
	on $X=\prod_{i\in I} X_i$ is composed by the cartesian products $\prod_{i\in I}B_i$,
	so that $\exists I_0\subset I$ finite such that
	$B_i\in \B_i$ for $i\in I_0$ and $B_i=X_i$ for $i\in I\setminus I_0$.
\end{theorem}

\begin{proof}Let us consider the topology $\T$ on $X$ generated by the following subbase:
	$$
	\S = \setdef{\pi^{-1}_i[B]}{B\in\B_i,\,i\in I}.
	$$
	We can prove that $\T=\T_X$ by using double inclusion.
	\begin{itemize}
		\item[$\boxed{\subseteq}$] By definition, every $S\in \S$ is open in $\T_X$, thus $\B\in \T_X$ $\Then$ $\T\subseteq \T_X$.
		\item[$\boxed{\supseteq}$] Let $\S_X$ be a subbase of $\T_X$. According to \defref{def:product-topology}:
		$$
			\S_X = \setdef{\pi^{-1}_j[A]}{A\in\T_j, j\in I}.
		$$
		Since $A\in \T_i$ and $\B_i$ is base for $\T_i$, $A$ can be expressed as union of
		base opens: $A=\bigcup_{B\in \B_i} B$. Therefore,
		$\pi^{-1}_i[A] = \pi^{-1}_i\left[\bigcup_{B\in \B_i} B\right] = \bigcup_{B\in \B_i} \pi^{-1}_i[B] = \bigcup_{S\in\S}S$.
		Since $\S$ is subbase for $\T$ ($\S\subset \T$) $\Then$ $\pi^{-1}_i[A]=\bigcup_{S\in\S}S\in \T$ $\Then$ $\T_X\subseteq \T$.
	\end{itemize}
	Let us define the family $\B_X$ as follows:
	$$
		\B_X = \setdef{\prod_{i\in I}B_i}{\left(B_i\in\B_i,\,\forall i\in I_0\right)\,\land\,\left(B_i=X_i,\,\forall i\in I\setminus I_0\right)}.
	$$
	We want to prove that all the elements of $\B_X$ are in the topology $\T$, hence of the product topology $\T_X$.
	Then, if we prove that $\B_X$ is a base for $\T$, we will have proved that $\B_X$ is a base for $\T_X$.

	Let us first prove that $B\in \T_X$, $\forall B\in \B_X$. Let $B\in\B_X$, according to \lemref{lem:projection} and the definition of the elements of $\B_X$:
	$B=\bigcap_{i\in I}\pi^{-1}_i[\pi_i[B]] = \left(\bigcap_{i\in I_0}\pi^{-1}_i[B_i]\right)\cap \left(\bigcap_{i\in I\setminus I_0}\pi^{-1}_i[X_i]\right)=\left(\bigcap_{i\in I_0}\pi^{-1}_i[B_i]\right)\cap X=\bigcap_{i\in I_0} \pi^{-1}_i[B_i]$.
	Therefore, $B$ is an intersection of elements of $\S$, hence $B\in \T = \T_X$. Therefore, $\B_X\subset \T_X$.

	To prove that $\B_X$ is a base for $\T$ (and, therefore, for $\T_X$) we are going to prove that $\B_X$ fulfills
	the conditions of \thref{th:1-3}:
	\begin{enumerate}
		\item Let us prove that $\bigcup\B_X = X$:
		\begin{itemize}
			\item[$\boxed{\subseteq}$] Let $x\in \bigcup\B_X$ $\Then$ $x\in \left(\prod_{i\in I_0} B_i\right)\times\left(\prod_{i\in I\setminus I_0}X_i\right)\subset\prod_{i\in I} X_i = X$ $\Then$ $x\in X$ $\Then$ $\bigcup \B_X\subset X$.
			\item[$\boxed{\supseteq}$] Let $C\subset \prod_{i\in I}X_i$ $\Then$ $C=\prod_{i\in I}C_i$ with $C_i\subset X_i$.
			Since $\B_i$ is base for $X_i$, $C_i=\bigcup_{B_i\in \B_i} B_i$, $\forall i\in I$.
			Hence, $C=\prod_{i\in I}\left(\bigcup_{B_i\in \B_i} B_i\right) = \bigcup \left(\prod_{i\in I} B_i\right) \subset \bigcup \B_X$ $\Then$ $X\subset \bigcup \B_X$.
		\end{itemize}
		\item Now consider $U,V\in \B_X$, and $x\in U\cap V$. Then $\exists I_0, I_1 \subset I$ such that
		$U=\bigcap_{i\in I_0}\pi^{-1}_i[B_i]$ and $V=\bigcap_{i\in I_1}\pi^{-1}_i[B_i']$.
		We can define $I^*=I_0\cup I_1$, and complete the definitions $B_i=X_i,\,\forall i\in I_1\setminus I_0$ and
		$B_i'=X_i,\,\forall i\in I_0\setminus I_1$.
		Thus,
		\begin{align*}
			U\cap V &= \left(\bigcap_{i\in I^*}\pi^{-1}_i[B_i]\right)\cap\left(\bigcap_{i\in I^*}\pi^{-1}_i[B_i']\right)\\
			&= \bigcap_{i\in I^*}\left(\pi^{-1}_i[B_i]\cap\pi^{-1}_{i}[B_i']\right)\\
			&= \bigcap_{i\in I^*}\pi^{-1}_i[B_i\cap B_i'].
		\end{align*}
		Since $x\in (U\cap V)$ $\Then$ $\pi_i(x)\in\pi_i(U\cap V)$ $\overset{\lemref{lem:projection}}{\Then}$
		$\pi_i(x)\in (B_i\cap B_i'),$ with $(B_i\cap B_i')\in \B_i$ by definition.
		Therefore, $\exists B_i''\in \B_i$ such that $\pi_i(x)\in B_i''\subset (B_i\cap B_i')$.
		This means that $x_i\in \pi^{-1}_i(x) \in \pi^{-1}_i[B_i'']\subset \pi^{-1}_i[B_i\cap B_i']$,
		so:
		$$
		 x\in \bigcap_{i\in I^*}\pi^{-1}_i[B_i'']\subset \bigcap_{i\in I^*}\pi^{-1}_i[B_i\cap B_i'] = U\cap V.
		$$
		With $\bigcap_{i\in I^*}\pi^{-1}_i[B_i'']\in \B_X$ by definition.
		Thus, the conditions of \thref{th:1-3} are fulfilled.
	\end{enumerate}
	With this, we have proved that $\B_X$ is a base for $\T$, and therefore for $\T_X$.
\end{proof}

We can readily note that in the particular case of a finite product of topological spaces,
the \thref{th:product-topology-base} ensures there is a base for the product topology
on $X_1\times\ldots X_n$ that is composed by subsets of the form $B_1\times\ldots\times B_n$,
where $B_i\in \B_i$, and $\B_i$ is a base for $(X_i, \T_i)$, $\forall i\in\{1,\ldots,n\}$.

\begin{remark}
	Although the product of open sets $X_i$ are a base for the product topology in the case of
	finite products, they are not the only open sets in the topology.
	For instance, it can be shown (and this will become trivial to prove in next chapters) that
	a circle without the border is an open set in $\mathbb{R}^2$, but it cannot be expressed as
	the cartesian product of two open intervals in $\mathbb{R}$, since that would be an open
	rectangle. How is this possible? Because the circle without the border can be expressed as the union
	of open rectangles, which are open sets in $\mathbb{R}^2$.
\end{remark}

\subsubsection{Box topology}\label{subsec:box-topology}

Having introduced the product topology, we can now introduce the \emph{box topology},
which is the topology in the cartesian product of topological spaces generated by the
product of open sets in each of the topological spaces.

\begin{definition}
	\label{def:box-topology}
	Let $\{(X_i, \T_i)\}_{i\in I}$ be a family of topological spaces, and let us consider
	the cartesian product $X=\prod_{i\in I} X_i$.
	The \textbf{box topology} (or \emph{box-product topology}) on $X$ (i.e., $\T_{X}$) is
	the set $X=\prod_{i\in I}X_i$ equipped with the topology $\T_X$ which is generated by the base:
	$$
		\B_X = \setdef{\prod_{i\in I}U_i}{U_i\in\T_i, i\in I}.
	$$
	The name of this topology comes from the fact that in $\mathbb{R}^n$ the base sets
	$\prod_{i\in I}B_i$ look like boxes.
	The box topological space is somtimes denoted as $\square_{i\in I}X_i$.
\end{definition}

Hereinafter, whenever we discuss the product of topological spaces
$\prod_{i\in I}X_i$ we will be implicitly referring to the (Tychonoff) product topology,
and not to the box topology unless otherwise stated.

Indeed, as we have seen so far, the Tychnoff product topology coincides with the box topology
in the case of a finite amount of factors, i.e. $X=X_1\times\ldots\times X_n$ (every
$X_i$ equipped with its own topology $\T_i$). However, when we have an infinite product
of topological spaces, the Tychnoff product topology has more desirable properties than
the box topology. For instance, if all the topological spaces $X_i$ are compact (see Chapter~\ref{ch:compactness}
for further details), then the Tychnoff product topology is always compact, whilst the
box topology is not necessarily compact. This is why, amongst other reasons, the Tychnoff
product topology is the most commonly used topology in the product of topological spaces.

\begin{definition}\label{def:usual-topology}
	Let us consider $\mathbb{R}$ and its \emph{usual} topology, i.e., the topology generated
	by the base $\B=\setdef{(a,b)}{a,b\in\mathbb{R}, a<b}$.
	Then, the \textbf{usual topology} on $\mathbb{R}^n$ is the Tychnoff product topology
	on $\mathbb{R}^n$, i.e. $\mathbb{R}^n$ equipped with the topology generated by the base
	$$
		\B_{\mathbb{R}^n}=\setdef{I_{(a_1,b_1)}\times\ldots\times I_{(a_n,b_n)}}{a_i,b_i\in\mathbb{R}, a_i<b_i,\,i\in\{1,\ldots,n\}}.
	$$
\end{definition}

\subsection{Subspace topology}\label{subsec:subspace-topology}

In this section we introduce the topological properties of a subset $Y$ of a
topological space $X$. In particular, we will be interested in the topology on $Y$
which is induced by the topology on $X$. This topology is called the \emph{subspace topology},
and it is also referred to as the \emph{relative topology} or \emph{induced topology}.

\begin{theorem}
	\label{th:subspace-topology}
	Let $(X, \T)$ be a topological space, and $Y\subset X$. Then, the family
	$\Trel{Y} = \setdef{U\cap Y}{U\in\T}$ is a topology on $Y$.
\end{theorem}
\begin{proof}
Let us confirm that $\Trel{Y}$ fulfils the conditions of \defref{def:topology}:
	\begin{enumerate}
		\item $\emptyset\in\Trel{Y}$, since $\emptyset\in\T$, and $\emptyset\in(\emptyset\cap Y)\in \Trel{Y}$. Besides,
		$Y\in \Trel{Y}$, since $X\in\T$ and $Y=(X\cap Y)\in \Trel{Y}$. Therefore, $\emptyset, Y\in \Trel{Y}$.
		\item Let $\F\subset \Trel{Y}$, such that $\F=\setdef{F\cap Y}{F\in \T}$. Let us consider the family of
		open sets associated to $\F$, i.e., $\F^*=\setdef{U\in\T}{(U\cap Y)\in \F}$. Consider
		$A=\bigcup_{F\in\F^*}F=\bigcup \F^*$, which by definition is an open set in $\T$.
		We can check that:
		$$
		A\cap Y = \left(\bigcup\F^*\right)\cap Y=\bigcup_{F\in \F^*}\underbrace{(F\cap Y)}_{\in\F}=\bigcup \F.
		$$
		Therefore, $\bigcup \F = (A\cap Y)$, and we know that $A=\bigcup \F^* \in \T$ $\Then$ $\bigcup \F\in \Trel{Y}$.
		\item Let $U, V\in \Trel{Y}$ $\Then$ $\exists A, B\in \T$ such that $U=(A\cap Y)$ and $V=(B\cap Y)$. Then,
		$(U\cap V) = (A\cap Y)\cap(B\cap Y) = (A\cap B)\cap Y$, with $(A\cap B)\in \T$ $\Then$ $(U\cap V)\in \Trel{Y}$.
	\end{enumerate}
\end{proof}


\begin{definition}
	\label{def:subspace-topology}
	The pair $(Y, \Trel{Y})$ is called a \textbf{subspace} of $(X, \T)$,
	and $\Trel{Y}$ is called the \textbf{subspace} (or \textbf{relative}, or \textbf{induced}) {topology} on $Y$.
\end{definition}

Following the previous results, we can now proceed and introduce the base and subbase
for the subspace topology induced by their counterparts in the original topological space.
These results can be easily proved by using the definitions of base and subbase, and
the definition of subspace topology, as we will see in the following lemma.

\begin{lemma}
	\label{lem:subspace-topology-base}
	Let $(X, \T)$ be a topological space, and $\B$ and $\S$ a base and subbase for $\T$, respectively.
	Then, the following statements hold true:
	\begin{itemize}
		\item $\Rel{\B}{Y}=\setdef{B\cap Y}{B\in\B}$ is a base for $\Trel{Y}$.
		\item $\Rel{\S}{Y}=\setdef{S\cap Y}{S\in\S}$ is a subbase for $\Trel{Y}$.
	\end{itemize}
\end{lemma}

\begin{proof}
	Let us proceed in order:
	\begin{itemize}
	\item We are going to use the definition of base for a topology in order to prove that $\Rel{\B}{Y}$  is base for the relative topology $\Trel{Y}$. By definition, $\forall U\in\Trel{Y}$ is of the form $U = A\cap Y,$ with $A\in\T$. We know $\B$ is base for $\T$, hence $\exists \F\subset \B$ such that $A=\bigcup \F$, so $U=(\bigcup \F)\cap Y$. Consider $\F^*=\setdef{U\cap Y}{U\in\F}\subset\Rel{\B}{Y}$, therefore:
	$$
	  U=\underbrace{\left(\bigcup \F\right)}_{\in\B}\cap Y=\bigcup_{F\in \F}F\cap Y=\bigcup \F^*
	$$
	Therefore, $\forall U\in\Trel{Y}$ we can find $\F^*\subset \Rel{\B}{Y}$ such that $U=\bigcup \F^*$, which means that $\Rel{\B}{Y}$ is base for $\Trel{Y}$.
	\item Let $\B$ be the base induced by $\S$, i.e. $\B=\setdef{\bigcap S}{S\in\S}$. We can check that $\Rel{\B}{Y} = \setdef{B\cap Y}{B\in\B}$ (which we have already proved to be a base for $\Trel{Y}$) is the base induced by $\Rel{\S}{Y} =\setdef{S\cap Y}{S\in\S}$, i.e. $\B_{\S_Y}=\setdef{\bigcap S'}{S'\in\Rel{\S}{Y}}$. Let us prove this:
		\begin{align*}
		\B_{\S_Y}=&\setdef{\bigcap S'}{S'\in\Rel{S}{Y}}=\setdef{\bigcap (S\cap Y)}{S\in\S}\\
		=&\setdef{\left(\bigcap S\right)\cap Y}{S\in\S}=\setdef{B\cap Y}{B\in\B}\\
		=&\Rel{\B}{Y}
		\end{align*}
	This means that $\Rel{\S}{Y}$ is subbase for $\Trel{Y}$.
	\end{itemize}
\end{proof}

In what follows, we will always consider the subsets of topological spaces as topological spaces equipped with the subspace topology, unless otherwise stated.
This is coherent, given that if we consider $X$ to be a topological space, and we have $Z\subset Y\subset X$, then the relative topology on $Z$ with respect to $X$ is the same as the relative topology on $Z$ with respect to $Y$ when $Y$ is equipped with the relative topology with respect to $X$.

Regarding the product topology, we have the following result about the coherence of the product of topological spaces with the family of subspaces of each of the factors.

\begin{theorem}
	\label{th:product-subspace-topology}
	Let $\{(X_i, \T_i)\}_{i\in I}$ be a family of topological spaces, with $X=\prod_{i\in I} X_i$,
	$\{Y_i\}_{i\in I}$ be a family with $Y_i\subset X_i$, $\forall i\in I$, and $Y=\prod_{i\in I}Y_i\subset X$.
	Considering each $Y_i$ as a topological space equipped with the subspace topology
	induced by $X_i$, then the product topology on $Y$ is the same as the subspace topology induced by $X$ on $Y$.
\end{theorem}

\begin{proof}
	By definition (see \defref{def:product-topology} and \thref{th:product-topology-base}), the product topology on $Y=\prod_{i\in I}Y_i$ is generated by a base $\B_Y$ such that:
	$$
	\B_Y=\setdef{\prod_{i\in I_0}D_i\times \prod_{i\in(I\setminus I_0)}Y_i}{D_i\in\B_{Y_i},\,I_0\subset I}.
	$$
	where $\B_{Y_i}=\setdef{B_i\cap Y_i}{B_i\in \B_i}$ is the base induced by $\B_i$ on $Y_i$.

	Also by definition (see \defref{def:subspace-topology}), the subspace topology on $Y\subset X$ is generated by a base $\Rel{\B}{Y}$ such that:
	\begin{align*}
	\Rel{\B}{Y}=&\setdef{B\cap Y}{B\in\B_X}\\
	=&\setdef{\prod_{i\in I}B_i\cap \prod_{i\in I}Y_i}{\left(B_i\in\B_{X_i},\forall i\in I_0\subset I\right)\land\left(B_i=X_i, \forall i\in I\setminus I_0\right)}.
	\end{align*}
	with $\B_X$ being the base for the product topology on $X$.

	\noindent{\boxed{\subseteq}} Let us consider $F\in \B_{Y}$, then:
	\begin{align*}
	F=&\prod_{i\in I_0} D_i \times \prod_{i\in(I\setminus I_0)}Y_i=\prod_{i\in I_0} (B_i\cap Y_i) \times \prod_{i\in(I\setminus I_0)}(X_i\cap Y_i)=\prod_{i\in I}(A_i\cap Y_i)
	\end{align*}
	with $A_i$ being $B_i\in \B_{X_i}$ if $i\in I_0$, and $X_i$ if $i\in I\setminus I_0$.
	Thus, we have proved that $F\in\B_Y$ $\Then$ $F\in\Rel{\B}{Y}$, i.e. $\T_Y\subset \Trel{Y}$.

	\noindent{\boxed{\supseteq}} Let us consider $F\in \Rel{\B}{Y}$, then:
	\begin{align*}
		F=&\prod_{i\in I}B'_i\cap \prod_{i\in I}Y_i = \left(\prod_{i\in I_0} B_i\times \prod_{i\in I\setminus I_0} X_i\right)\cap \prod_{i\in I}Y_i\\
		=&\left(\prod_{i\in I_0} B_i\times \prod_{i\in I\setminus I_0} X_i\right)\cap \left(\prod_{i\in I_0} Y_i\times \prod_{i\in I\setminus I_0} Y_i\right)\\
		=&\prod_{i\in I_0} (B_i\cap Y_i) \times \prod_{i\in I\setminus I_0} (X_i\cap Y_i) = \prod_{i\in I_0} D_i \times \prod_{i\in I\setminus I_0} Y_i
	\end{align*}
	with $D_i\in \B_{Y_i}$, $I_0\subset I$. Thus, we have proved that $F\in\Rel{\B}{Y}$ $\Then$ $F\in \B_Y$, i.e. $\Trel{Y}\subset \T_Y$.

	This means that $\T_Y=\Trel{Y}$, i.e. the product topology on $Y$ is the same as the subspace topology induced by $X$ on $Y$.
\end{proof}

\subsection{Hausdorff spaces}\label{subsec:hausdorff-spaces}

In his original definition of topological space\footnote{Hausdorff, Felix (1914). \emph{Grundz\"{u}ge der Mengenlehre}, Leipzig: Veit, ISBN 978-0-8284-0061-9. Reprinted by Chelsea Publishing Company in 1944, 1949 and 1965}, Hausdorff introduced the concept in terms of neighbourhoods, requiring a certain property which seemed natural at that time and that is not present in the modern definition.
This is because there exist topologies which do not satisfy this property, as was discovered much later.
Nevertheless, such a property is useful to distinguish between topological spaces which do satisfy it and those which do not.
A generalisation of such a restriction gave rise to the concept of the \emph{separation axioms}, which are a set of properties that a topological space can have or not.
These axioms are sometimes called \emph{Tychnoff separation axioms}, and denoted with the letter $T$ after the German Trennungsaxiom (``separation axiom''), and a numerical subscript (e.g. $T_0$, $T_1$, $T_2$, etc.) to indicate a stronger and stronger separation property.
Despite their names, the separation axioms are not fundamental axioms as those of set theory, but rather defining properties of topological spaces.
In what follows, we are going to introduce three of such axioms, which are the most commonly used in the literature.
However, a more extensive list of separation axioms can be found elsewhere\footnote{Willard, Stephen (1970). \emph{General topology}. Reading, Mass.: Addison-Wesley Pub. Co. ISBN 0-486-43479-6.} if the reader is interested.

\begin{definition}
	Let $(X, \T)$ be a topological space. We say that the space fulfils the separation axiom (or, simply, is a):
	\begin{itemize}
		\item[$T_0$:] If $x,y\in X$, and $x\neq y$, $\exists U\in\T$\,:\,$(x\in U) \land (y\notin U)$, or $(x\notin U) \land (y\in U)$.
		\item[$T_1$:] If $x,y\in X$, and $x\neq y$, $\exists U\in\T$\,:\,$(x\in U)\land (y\notin U)$.
		\item[$T_2$:] If $x,y\in X$, and $x\neq y$, $\exists U,V\in\T$\,:\,$(x\in U)\land (y\in V) \land (U\cap V=\emptyset)$.
	\end{itemize}
	$T_0$ is called the \emph{Kolmogorov axiom}, $T_1$ is called the \emph{Fr\'{e}chet axiom}, and $T_2$ is called the \emph{Hausdorff axiom}.
	A topological space which fulfils the Hausdorff axiom is called a \textbf{Hausdorff space}, and the same applies to the other axioms, i.e. a $T_0$ space is a \emph{Kolmogorov space}, and a $T_1$ space is a \emph{Fr\'{e}chet space}.
\end{definition}

\begin{remark}
	In plain English, a topological space is a Hausdorff space if every pair of points in the space can be separated by disjoint open sets.
	We can readily check that metric spaces and ordered spaces are Hausdorff spaces.
\end{remark}

\begin{example}
	Let $\mathcal{M}$ be a \emph{metric space}\footnote{The concept of metric space will be introduced in Chapter~\ref{ch:metric-spaces}} equipped with the topology $\T_d$ induced by the open balls $B_\epsilon(x)=\setdef{y\in \mathcal{M}}{(x\in\mathcal{M})\land (d(x,y)<\epsilon)}$, where $d$ is the \emph{metric} of $\mathcal{M}$. Then, $\forall x,y \in \mathcal{M}$ we can always find $B_{\epsilon/2}(x)$ and $B_{\epsilon/2}(y)$ such that $\epsilon=d(x,y) > 0$.
	Therefore, $x\in B_{\epsilon/2}(x)\in\T$ and $y\in B_{\epsilon/2}(y)\in \T_d$, such that $B_{\epsilon/2}(x)\cap B_{\epsilon/2}(y)=\emptyset$, so $\mathcal{M}$ is a Hausdorff space (or simply $T_2$).
\end{example}

It is straightforward to prove that a every $T_2$ space is also $T_1$, and that every $T_1$ space is also $T_0$. We leave this simple proof to the reader.

\begin{remark}
	A topological space will fail the $T_0$ axiom \iff there are two distinct points $x,y\in X$ such that $\forall U\in\T: x\in U$, then $y\in U$. If this is the case, we say that $x,y$  are \emph{topologically indisitinguishable}, or that they are \emph{topologically equivalent}, given that any topological property of $x$ will also be a topological property of $y$. This is why we often find the $T_0$ axiom expressed as: ``$X$ is $T_0$ if any two distinct points in $X$ are topologically distinguishable.''
\end{remark}

\begin{example}
	Let $(X,\T)$ be a topological space, with the trivial topology $\T=\{X,\emptyset\}$.
	Then, $\forall x,y\in X$, with $x\neq y$, we have that the only open set containing $x$ is $X$, and the only open set containing $x$ or $y$ is $X$ itself. Thus, $x$ and $y$ are topologically equivalent, and $(X,\T)$ is not $T_0$.
\end{example}

\begin{example}
	Let $X=\mathbb{R}^2$, with the topology $\T=\setdef{B_\epsilon(x)}{x\in\mathbb{R}^2}$, induced by open balls $B_\epsilon(x)=\setdef{y\in\mathbb{R}^2}{d(x,y)=|x_1-y_1|<\epsilon}$ induced by the pseudo-metric $d$.
	Then, all points in a vertical line on the plane, $\setdef{x\in \mathbb{R}^2}{x_1=c,\,c\in \mathbb{R}}$, are topologically equivalent.
	Each of these lines comprise a topological unit.
\end{example}

\begin{example}
	A simple example of a topological space which is $T_0$ but not $T_1$ is the one defined on $X=\{0,1\}$ with the topology $\T=\{\emptyset,\{1\},\{0,1\}\}$, also called \emph{Sierpi\'{n}ski space}. We can see how the two points are topologically distinguishable, but there is no open set containing $\{0\}$ but not $\{1\}$, or \emph{vice versa}. Thus, the space is $T_0$ but not $T_1$.
\end{example}

Now that we have introduced the most commonly used separation axioms, and we have gained a bit of intuition about them with the previous examples, let us introduce a result which shows that the separation axioms are preserved when we consider subspaces of topological spaces.

\begin{theorem}
	\label{th:subspace-separation-consistency}
	Let $(X, \T)$ be a topological space, and $Y\subset X$ a subspace of $X$ with the subspace topology $\Trel{Y}$.
	Then, if $X$ is $T_i$ $\Then$ $Y$ is also $T_i$, $i\in\{0,1,2\}$.
\end{theorem}
\begin{proof}
	Let us prove the case $T_2$.
	Consider $x,y\in Y\subset X$, with $x\neq y$.
	Since $X$ is $T_2$, $\exists U,V\in\T$ such that $(x\in U)\land (y\in V)\land (U\cap V=\emptyset)$.
	Given that $U,V\in\T$ $\Then$ $(U\cap Y)\in \Trel{Y}$ and $(V\cap Y)\in \Trel{Y}$.
	Also, $x\in (U\cap Y)$ and $y\in (V\cap Y)$, and $(U\cap Y)\cap (V\cap Y)=\emptyset$.
	Therefore, $\exists U', V'\in \Trel{Y}$\,:\,$(x\in U')\land (y\in V')\land (U'\cap V'=\emptyset)$ $\Then$ $Y$ is $T_2$.
	The same reasoning can be applied to prove the result for $i=0,1$.
\end{proof}

\begin{theorem}
	\label{th:product-topology-separation-consistency}
	Let $\{X_i\}_{i\in I}$ be a family of topological spaces, such that \textbf{$X=\prod_{i\in I}X_i\neq \emptyset$}.
	Then, $X$\footnote{With the product topology} is a $T_j$ space $\iff$ $\forall i\in I$, $X_i$ is a $T_j$ space, for $j\in\{0,1,2\}$.
\end{theorem}
\begin{proof}
	As we did with the previous result in \thref{th:subspace-separation-consistency}, we are going to prove the case $T_2$ only, since the other cases can be proved in a similar way.
	Given that we have to prove an $\iff$ statement, we are going to prove both implications:
	\begin{itemize}
	\item[$\boxed{\Rightarrow}$] Let $X$ be a $T_2$ space, and let $u_j, v_j \in X_j$ two distinct points, such that they are the $j$-th coordinate of the points $u,v\in X$, respectively, i.e. $\pi_j(u)=u_j$ and $\pi_j(v)=v_j$. Since $X$ is $T_2$, $\exists U, V\in \T_X$ such that $(u\in U)\land (v\in V)\land (U\cap V=\emptyset)$. Therefore,
	$$
	U\cap V= \prod_{i\in I}U_i\cap\prod_{i\in I}V_i=\prod_{i\in I}(U_i\cap V_i)=\emptyset,
	$$
	with $U_i, V_i$ satisfying the conditions of \thref{th:product-topology-base}.
	This means that $(U_i\cap V_i)=\emptyset$, $\forall i\in I$. In particular, $(U_j\cap V_j)=\emptyset$.
	Thus, $\forall u_j, v_j\in X_j$, $\exists U_j, V_j \in \T_{X_j}$ such that $(u_j\in U_j)\land (v_j\in V_j)\land (U_j\cap V_j=\emptyset)$, i.e. $X_j$ is $T_2$.
	\item[$\boxed{\Leftarrow}$] Let $X_i$ be a $T_2$ space $\forall i\in I$. Consider $u,v\in X$ two distinct points, $\exists j\in I$ so that $X_j\ni \pi_j(u) = u_j \neq v_j=\pi_j(v)\in X_j$. Then, $\exists U,V \in \T_j$ such that $(u_j\in U)\land (v_j\in V)\land (U_j\cap V_j)=\emptyset$.
	Using the definition of cartesian projection \defref{def:projection}:
	\begin{align*}
		u_j\in U &\Then u_j\in\pi^{-1}_j[U]=\setdef{x\in X}{x_j\in U}\\
		v_j\in V &\Then v_j\in\pi^{-1}_j[V]=\setdef{x\in X}{x_j\in V}
	\end{align*}
	Therefore, $\nexists x\in X$ such that $x\in \pi^{-1}_j[U]$ and $x\in \pi^{-1}_j[V]$,
	because that would require $x_j\in U$ and $x_j\in V$, but $U\cap V=\emptyset$.
	If we call $U'=\pi^{-1}_j[U]$ and $V'=\pi^{-1}_j[V]$, it is clear that $U', V'\in \T_X$ according to the definition of product topology \defref{def:product-topology}, hence $\exists U', V' \in \T_X$: $(u\in U')\land (v\in V')\land (U'\cap V'=\emptyset)$, i.e. $X$ is $T_2$.
	\end{itemize}
	This proves that $X$ is $T_2$ $\iff$ $X_i$ is $T_2$, $\forall i\in I$. The same reasoning can be applied to prove the result for $j=0,1$.
\end{proof}

\section{Topological concepts}\label{sec:topological-concepts}

Thus far, we have been introduced to the concept of topological space, and the formal definition of \emph{surrounding} (or \emph{neighbourhood}) of a point in a topological space without the need of resorting to the concept of distance, besides the concepts of \emph{base} and \emph{subbase}, which are useful to define topologies in a more compact way.
Later, we have taken these concepts, and their related results, to a more general realm as is the case of product spaces.
Finally, we have introduced a set of properties that a topological space can have, called \emph{separation axioms}, which are useful to distinguish between topological spaces.
In what follows, we are going to introduce and prove some more results which are core to the study of topological spaces, and which will be useful in this and the following chapters.

\subsection{Closed sets}\label{subsec:closed-sets}

The concept of \emph{closed set}, which we are about to introduce, plays a very important role in the study
of topological spaces.
As the name suggests, it formalises the concept of being ``closed'' as opposed to being ``open'', which we have already introduced before.
We will later exploit this definition to develop a more intuitive characterisation of the separation axioms $T_1$ and $T_2$.

\begin{definition}
	\label{def:closed-set}
	Let $(X,\T)$ a topological space, and $A\subset X$.
	The subset $A$ is said to be \textbf{closed}, i.e. $A$ is a \textbf{closed subset}, in $X$ if $A^\complement=(X\setminus A)\subset X$ is an open subset.
\end{definition}

A straightforward result of this definition is that closed sets in a topological space satisfy \emph{dual properties} to those of open sets.

\begin{lemma}
	\label{lem:toplogy-closed-sets}
	Let $(X, \T)$ be a topological space. Then, the following statements hold true:
	\begin{enumerate}
		\item $X$ and $\emptyset$ are closed sets in $X$.
		\item Let $\{A_i\}_{i\in I}$ be a family of closed sets in $X$, then $\bigcap_{i\in I} A_i$ is also a closed set in $X$.
		\item Let $A, B\subset X$ be two closed sets in $X$, then $(A\cup B)$ is also closed in $X$.
	\end{enumerate}
\end{lemma}

\begin{proof}
	Let us proceed in order:
	\begin{enumerate}
		\item $X$ is closed given that $X^\complement=\emptyset\in \T$. The same applies to $\emptyset$, because $\emptyset^\complement=X\in \T$.
		Therefore, $X$ and $\emptyset$ are closed sets in $X$.
		\item Consider the family $\{A_i\}_{i\in I}$, with $A_i$ closed sets in $X$, i.e. $A_i^\complement\in \T$ $\Then$ $\left(\bigcup_{i\in I}A_i^\complement\right)\in \T$, by hypothesis. Consider now $A=\bigcap_{i\in I}A_i$ $\Then$ $A^\complement = \left(\bigcap_{i\in I} A_i \right)^\complement=\bigcup_{i\in I}A_i^\complement$, by using the De Morgan's laws. Therefore, $A^\complement\in \T$ $\Then$ $A$ is a closed set in $X$.
		\item Consider $A,B\subset X$ two closed sets, which means that $A^\complement, B^\complement\in \T$ $\Then$ $(A^\complement\cap B^\complement)\in \T$, by hypothesis. Consider now $D=(A\cup B)$, hence $D^\complement = (A\cup B)^\complement = (A^\complement \cap B^\complement)\in \T$, by using the De Morgan's laws. Therefore, $D^\complement\in \T$ $\Then$ $D$ is a closed set in $X$.
	\end{enumerate}
	With this, it is proved that closed sets in a topological space satisfy dual properties to those of open sets.
\end{proof}

The following result relates closed sets in a topological space $X$, and closed sets in the subspace topology for $Y\subset X$:

\begin{lemma}
	\label{lem:closed-sets-subspace-topology}
	Let $(X, \T)$ be a topological space, and $Y\subset X$. The closed sets of $\Trel{Y}$ are the sets of the form $Y\cap C$, with $C$ closed in $X$.
\end{lemma}
\begin{proof}
	Recall that the relative topology on $Y$ is generated by the base $\Rel{\B}{Y}=\setdef{B\cap Y}{B\in\B}$, with $\B$ being a base for the topology $\T$ on $X$. The closed sets in $Y$ are going to be precisely the complements of such sets, i.e., $F=Y\setminus(Y\cap B)$, with $B\in\B$. Now:
	$$
	F=Y\setminus(Y\cap B)=\setdef{x\in X}{(x\in Y)\land(x\notin B)}=Y\cap B^\complement
	$$
	where $B^\complement$ are closed sets in $X$ $\Then$ $F=Y\cap C$, with $C$ closed in $X$.
\end{proof}

With the help of these new concepts, we can understand in a more intuitive way what a $T_1$ space really means.
This can be seen in the following result:

\begin{theorem}
	\label{th:t1-closed-points}
	A topological space is $T_1$ \iff\ every singleton is a closed set.
\end{theorem}
\begin{proof}
	Let us prove both implications:
	\begin{itemize}
			\item[$\boxed{\Rightarrow}$] Let $(X,\T)$ be a $T_1$ space, then $\forall x,y \in X$ such that $x\neq y$, $\exists U\in\T$ with $(y\in U)\land (x\notin U)$. This means that $y\in U\subset (X\setminus\{x\})$ $\Then$ $(X\setminus\{x\})$ is neighbourhood of all its points, because $\forall y\in (X\setminus\{x\})$, $\exists U\in \T$ such that $y\in U\subset (X\setminus\{x\})$ $\overset{\text{\thref{th:open-subset-neighbourhood}}}{\Then}$ $X\setminus\{x\}$ is open, so $\{x\}$ is closed in $X$.
		\item[$\boxed{\Leftarrow}$] Given that all singletons are closed,  then $(X\setminus\{x\})\in \T,\,\forall x\in X$ by definition. Thus, $\forall x,y\in X$ such that $x\neq y$, obviously $y\in X\setminus\{x\}\,\land\,x\notin (X\setminus\{x\})$ $\Then$ $\exists U\in \T$ (e.g., $U=X\setminus\{x\}$) such that $y\in U$ and $x\notin U$ $\Then$ $X$ is $T_1$.
	\end{itemize}
\end{proof}

\begin{remark}
	Note that if $X$ is a $T_1$ space, then \emph{every finite subset}\footnote{Formally, a set $A$ is \textbf{finite} if there is a bijection $f:A\rightarrow\{1,\dots,n\}$ for some $n\in \mathbb{N}$, so that $A$ has $n$ elements. The number of elements of $A$ is called the \textbf{cardinality} of $A$, and it is denoted by $|A|$.} $Y\subset X$ is also closed, because $Y=\bigcup_{y\in Y}\{y\}$, and $\{y\}$ is closed $\forall y\in Y$.
	Hence, a finite space is a $T_1$ space \iff\ it is discrete, given that all its subsets are closed, so all are open.
\end{remark}

%\begin{Example}{Co-finite is $T_1$ but not $T_2$}{exa:cofinite-not-t2}
\begin{example}
	In a previous example we showed that the Sierpi\'{n}ski space was $T_0$ but not $T_1$. We can show something analogous for the co-finite topology:
	$$
	\T_\complement=\setdef{A\subset X}{A=\emptyset\lor A^\complement\text{ is finite}}.
	$$
	We can prove that $\T_\complement$ is $T_1$ but not $T_2$. As a direct consequence of its definition, the only closed sets in the co-finite topology are the finite sets, therefore all singletons are closed, so $(X, \T_\complement)$ is a $T_1$ space. What about $T_2$? If $X$ is not finite, for any $A,B\in \T_\complement$ we have:
	\begin{align*}
		A\in\T_\complement\ \Then&\ A=\emptyset\,\lor\,\exists n\in \mathbb{N}\,:\,A^\complement=\{x_1,\dots,x_n\},\\
		B\in\T_\complement\ \Then&\ B=\emptyset\,\lor\,\exists m\in \mathbb{N}\,:\,B^\complement=\{y_1,\dots,y_m\},
	\end{align*}
	Therefore, $\exists x\in X$ such that $(x\in A)\land(x\in B)$, always $\Then$ $\nexists A,B\in \T_\complement$, such that $x\in A$ and $y\in B$, with $A\cup B=\emptyset$ $\Then$ $(X,\T_\complement)$ is not $T_2$.
\end{example}

Another interesting result that can be derived from the definition of closed set is a way to characterise $T_2$ spaces.

\begin{theorem}
	\label{th:t2-diagonal}
	A topological space $(X,\T)$ is a $T_2$ space \iff\ the diagonal set $\Delta=\setdef{(x,x)\in X^2}{x\in X}$ is closed in $X^2=X\times X$
\end{theorem}
\begin{proof}
Let us prove both implications:
\begin{itemize}
	\item[$\boxed{\Rightarrow}$] Let $X$ be $T_2$, then $\exists U,V\in \T$, such that $(x\in U)\land (y\in V)$, and $(U\cap V)=\emptyset$ $\Then$ $(x,y)\in (U\times V)\in\T_{X^2}$. With this same argument we can guarantee that $\forall (x,y)\in (X^2\setminus\Delta)$, $\exists F\in \T_{X^2}$ (e.g., $F=U\times V$) such that $(x,y)\in F\subset (X^2\setminus \Delta)$ $\Then$ ($X^2\setminus \Delta)$ is neighbourhood of all its points, so it is open in $X^2$ $\Then$ $\Delta$ is closed.
	\item[$\boxed{\Leftarrow}$] Let $\Delta$ be closed in $X^2$, and consider $x,y\in X,\,(x\neq y)$ $\Then$ $(x,y)\in (X\setminus\Delta)\in\T_{X^2}$ ($\Delta$ closed $\Then$ $\Delta^\complement$ open).
	Thus, $\exists F=(U\times V)\in \T_{X^2}$ with $U,V\in \T$, such that $(x,y)\in (U\times V)\subset (X\setminus \Delta)$ $\Then$ $(x\in U) \land (y\in V) \land (U\cap V)=\emptyset$ necessarily, since $(U\times V)\subset(X\setminus \Delta)$ $\Then$ $\setdef{(x,x)}{x\in X}\not\subset (U\times V)$ $\Then$ $\nexists x\in (U\cap V)$. Therefore $X$ is $T_2$.
\end{itemize}
\end{proof}

\subsection{Interior and closure}\label{subsec:interior-closure}

Up to here, we have introduced the concept of open set purely based on the capacity of being a neighbourhood of all its points. Later, we came up with the definition of closed set, which is the dual concept (\emph{complement}) of open set. Keeping these two in mind, it seems clear that there are certain points which seem to be \emph{enclosing}, or surrounding, the \emph{interior} part of a set. And, this is precisely what we aim to define with precision in this section. We will start with the definitions of \emph{interior} and \emph{closure} of a set, and then we will introduce some results which will help us to understand the relation between these concepts. Finally, we will see how this extends to product spaces.

\begin{definition}
	\label{def:interior-closure}
	Let $(X,\T)$ be a topological space, and consider $A\subset X$.
	The \textbf{interior} of $A$ is the union of all subsets of $A$ which are open in $X$:
	\[\mathrm{int}(A)=\interior{A}=\bigcup\setdef{U}{(U\subset A)\,\land\,(U\in\T)}.\]
	The \textbf{closure} of $A$ is the intersections of all subsets of $X$ which contain $A$ and are closed in $X$:
	\[\mathrm{cl}(A)=\closure{A}=\bigcap\setdef{C}{(A\subseteq C)\,\land\,(C^\complement\in\T)}.\]
	The elements (points) of $\interior{A}$ are called \textbf{interior points} of $A$, and the elements of $\closure{A}$ are called \textbf{closure points} (or \textbf{adherent points}) of $A$.
\end{definition}

\begin{remark}
	A straightforward result of this definition is that $\interior{A}$ is an open set, and $\closure{A}$ is a closed set, since the union of open sets is open, and the intersection of closed sets is closed, respectively.
	Indeed, it is also straightforward that $\interior{A}\subset A\subset \closure{A}$.
	What is more:
	\begin{itemize}
		\item[$\triangleright$] $\interior{A}$ is the largest open set contained in $A$, i.e., $\nexists U\in \T$: $\interior{A}\subset U\subset A$.
		\item[$\triangleright$] $\closure{A}$ is the smallest closed set containing $A$, i.e., $\nexists C$, with $C^\complement\in \T$: $A\subset C\subset \closure{A}$.
	\end{itemize}
\end{remark}
This remark leads us to the following result:
\begin{lemma}
	\label{lem:interior-closure-subsets}
	Let $(X, \T)$ a topological space, and consider $A, B\subset X$ such that $A\subset B$. Then, we can state the following:
	\[
		(\interior{A}\subset\interior{B})\,\land\,(\closure{A}\subset\closure{B}).
	\]
\end{lemma}
\begin{proof}
	On the one hand, $\interior{A}\subset A$, and $\interior{B}\subset B$, by definition $\Then$ $\interior{A}\subset A\subset B$. Therefore, $\interior{A}\subset\interior{B}$, because $\interior{B}$ is the largest open set contained in $B$, and $\interior{A}$ is an open set contained in $B$.

	On the other hand, $A\subset \closure{A}$, and $B \subset \closure{B}$, by definition $\Then$ following a similar argument $\closure{A}\subset \closure{B}$, because $\closure{A}$ is the smallest closed set containing $A$, and $\closure{B}$ is a closed set containing $A$, so that if $\closure{B}\subset \closure{A}$, then $\closure{B}$ would be the smallest closed set containing $A$, which is a contradiction.
\end{proof}
The following result is self-evident from the definition of interior and closure:
\begin{lemma}
	\label{lem:interior-open-closure-closed}
	Let $(X, \T)$ be a topological space, and consider $A\subset X$. Then, $\interior{A} = A$ \iff\ $A$ is open, and $\closure{A} = A$ \iff\ $A$ is closed.
\end{lemma}
The proof of the previous result is a trivial consequence of the definitions of interior and closure, which is why we leave it to the reader. The following result is an interesting and useful relation between the interior and the closure of the union and intersection of sets, which will be useful for later results:
\begin{lemma}
	\label{lem:interior-closure-union-intersection}
	Let $(X, \T)$ be a topological space, and consider $A,B\subset X$. Then, we can state the following:
	\begin{itemize}
		\item $\closure{A\cup B}=\closure{A}\cup\closure{B}$.
		\item $\Interior{A\cap B}=\interior{A}\cap\interior{B}$.
	\end{itemize}
\end{lemma}
\begin{proof} To prove the equality of sets, we are going to prove the inclusion in both directions:
	\begin{itemize}
		\item \begin{itemize}
				\item[$\boxed{\subseteq}$] $A\subseteq \closure{A}$, and $B\subseteq\closure{B}$ $\Then$ $A\subseteq (\closure{A}\cup\closure{B})$ and $B\subseteq (\closure{A}\cup\closure{B})$ $\Then$ $\closure{(A\cup B)}\subseteq \closure{(\closure{A}\cup\closure{B})}$ $\Then$ $\closure{(A\cup B)}\subseteq (\closure{A}\cup\closure{B})$.
				\item[$\boxed{\supseteq}$] $\closure{A}\subseteq \closure{(A\cup B)}$ and $\closure{B}\subseteq\closure{(A\cup B)}$ $\Then$ $(\closure{A}\cup\closure{B})\subseteq \closure{(A\cup B)}$.
			\end{itemize}
		\item \begin{itemize}
				\item[$\boxed{\subset}$] $(A\cap B)\subset A$, and $(A\cap B)\subset B$ $\Then$ $\Interior{(A\cap B)}\subset \interior{A}$ and $\Interior{(A\cap B)} \subset \interior{B}$ $\Then$ $\Interior{(A\cap B)}\subset (\interior{A}\cap\interior{B})$.
				\item[$\boxed{\supset}$] $(\interior{A}\cap\interior{B})\subset A$ and $(\interior{A}\cap\interior{B})\subset B$ $\Then$ $\Interior{(\interior{A}\cap\interior{B})}=(\interior{A}\cap\interior{B})\subset\Interior{(A\cap B)}$.
			\end{itemize}
	\end{itemize}
\end{proof}

The following result establishes the relation between the interior and the closure via the complement of a set:

\begin{theorem}
	\label{th:interior-closure-complement-relations}
	Let $(X,\T)$ be a topological space, and consider $A\subset X$. The following statements hold true:
	\begin{itemize}
		\item $\closure{X\setminus A}$ = $X\setminus\interior{A}$, i.e. $\mathrm{cl}(A^\complement)=\mathrm{int}^\complement(A)$.
		\item $\Interior{X\setminus A}$ = $X\setminus\closure{A}$, i.e. $\mathrm{int}(A^\complement)=\mathrm{cl}^\complement(A)$.
	\end{itemize}
\end{theorem}
\begin{proof}
	As usual in this type of proofs, we are going to prove the inclusion in both directions:
\begin{itemize}
	\item \begin{itemize}
			  \item[$\boxed{\subset}$] We know that $\interior{A}\subset A$ $\Then$ $(X\setminus A) \subset (X\setminus \interior{A})$. And, we know that $X\setminus \interior{A}$ is closed in $X$, so $(\closure{X\setminus A})\subset (X\setminus \interior{A})$.
			  \item[$\boxed{\supset}$] We also know that $(X\setminus A) \subset (\closure{X\setminus A})$, by definition $\Then$ $(X\setminus \closure{(X\setminus A)}) \subset (X\setminus(X\setminus A))$ $\Then$ $(X\setminus \closure{(X\setminus A)})\subset A$. Given that $X\setminus \closure{(X\setminus A)}$ is open, $(X\setminus \closure{(X\setminus A)})\subset \interior{A}$, and therefore $(X\setminus \interior{A})\subset (\closure{X\setminus A})$.
	\end{itemize}
	This proves that $\closure{X\setminus A}=X\setminus \interior{A}$.
	\item \begin{itemize}
			  \item[$\boxed{\subset}$] We have that $\Interior{X\setminus A}\subset(X\setminus A)$ by definition $\Then$ $(X\setminus (X\setminus A))\subset (X\setminus\Interior{(X\setminus A)})$ $\Then$ $A\subset (X\setminus\Interior{(X\setminus A)})$. Since the right hand side is closed, $\closure{A}\subset (X\setminus\Interior{(X\setminus A)})$ $\Then$ $\Interior{(X\setminus A)}\subset (X\setminus\closure{A})$.
			  \item[$\boxed{\supset}$] We know that $\closure{A}\supset A$ $\Then$ $(X\setminus \closure{A})\subset (X\setminus A)$. Since the left hand side is open, $(X\setminus \closure{A})\subset (\Interior{X\setminus A)}$.
	\end{itemize}
	This proves that $\Interior{X\setminus A}=X\setminus \closure{A}$.
\end{itemize}
\end{proof}

Up to this point we have the definition of what the interior and the closure of a set are, how these two behave under the subset relation, the conditions under which a set equals its interior or its closure, and some useful relations between the interior and the closure. The following result provides us with a way to characterise interior and closure (so-called adherent) points of a subset of a topological space:

\begin{lemma}
	\label{lem:interior-closure-points}
	Let $(X,\T)$ be a topological space, and consider $A\subset X$:
	\begin{itemize}
		\item $x\in\interior{A}$ \iff\ $A$ is neighbourhood of $x$.
		\item $x\in\closure{A}$ \iff\ $\forall U\subset X$ neighbourhood of $x$, $U\cap A\neq\emptyset$.
	\end{itemize}
\end{lemma}
In other words, $x$ is an interior point of $A$ \iff\ $x$ is surrounded by points of $A$, and $x$ is a closure point of $A$ \iff\ every neighbourhood of $x$ contains points of $A$.

\begin{proof}
	Let us proceed with the proof of both implications in both cases:
	\begin{itemize}
		\item \begin{itemize}
				  \item[$\boxed{\Rightarrow}$] Since $x\in\interior{A}$ $\overset{\defref{def:interior-closure}}{\Then}$ $x\in\bigcup\F$ with $\F=\setdef{F}{(F\subset A) \land (F\in\T)}$ $\Then$ $\exists$ $F\in \T$, such that $x\in F\subset A$ $\Then$ $A$ is neighbourhood of $x$.
				  \item[$\boxed{\Leftarrow}$] Say $A$ is neighbourhood of $x$ $\overset{\defref{def:neighbourhood}}{\Then}$ $\exists U\in\T$ such that $x\in U\subset A$. Given that $\interior{A}$ contains all such open sets, $x\in\interior{A}$.
		\end{itemize}
		\item \begin{itemize}
				  \item[$\boxed{\Rightarrow}$] Say $x\in\closure{A}$, and let $U\subset X$ be a neighbourhood of $x$ $\Then$ $\exists G\in \T$ such that $x\in G\subset U$. If $(A\cap G)=\emptyset$ $\Then$ $A\subset (X\setminus G)$ $\Then$ $\closure{A}\subset(X\setminus G)$ $\Then$ $x\not\in G$, which is a contradiction $\Then$ $(A\cap G)\neq\emptyset$.
				  \item[$\boxed{\Leftarrow}$] We know that for all $U$ neighbourhood of $x$, $U\cap A\neq \emptyset$. Let us assume that $x\in(X\setminus \closure{A})$ $\Then$ since $(X\setminus \closure{A})$ is open, $\exists G\in\T$ (in particular, $G=X\setminus \closure{A}$) such that $x\in (G\cap U)$ $\Then$ $x\in G$ and $(G\cap A)=\emptyset$. This is a contradiction, since $x\in U$ and $(U\cap A)\neq\emptyset$ by hypothesis. Therefore, $x\in\closure{A}$.
		\end{itemize}
	\end{itemize}
\end{proof}

\begin{remark}
	The interior points of $A$ are those with a neighbourhood contained in $A$, whilst the closure points of $A$ are those which are \emph{adhered} to $A$ in the sense that $A$ intersects every neighbourhood of such points.
\end{remark}

Having established the notions of interior and closure, and how they relate to each other, we can now proceed and generalise such concepts to product spaces, i.e. cartesian products of topological spaces with the Tychonoff product topology.

\begin{theorem}
	\label{th:closure-product-space}
	Let $\{X_i\}_{i\in I}$ be a family of topological spaces, and $\{Y_i\}_{i\in I}$ a family such that $Y_i\subset X_i$, $\forall i\in I$. Then, we can state the following:
	\[\closure{\prod_{i\in I}Y_i}=\prod_{i\in I}\closure{Y_i}\]
	In particular, the product of closed sets is also closed.
\end{theorem}

\begin{proof}Let $X=\prod_{i\in I}X_i$ equipped with the product topology $\T_X$.
We can proceed with the proof of the inclusion in both directions:
\begin{itemize}
\item[$\boxed{\subset}$] Let us take the complement of the inverse projection of the closures:
\[
	X\setminus\pi^{-1}_j[\closure{Y_j}]=X\setminus \closure{Y_j}\times \prod_{\substack{i\in I\\i\neq j}} X_i= (X_j\setminus\closure{Y_j})\times \prod_{\substack{i\in I\\i\neq j}} X_i = \pi^{-1}_j[X_j\setminus\closure{Y_j}]
\]
Given that $X_j\setminus \closure{Y_j}$ is open in $X_j$, $\pi^{-1}_j[X\setminus\closure{Y_j}]$ is open in $X$, which means that $\pi^{-1}_j[\closure{Y_j}]$ is closed in $X$. Given that $\prod_{i\in I}\closure{Y_i}=\bigcap_{j\in I}\pi^{-1}_j[\closure{Y_j}]$ $\Then$  $\prod_{i\in I}\closure{Y_i}$ is closed because is the intersection of closed sets in $X$ (see~\thref{lem:toplogy-closed-sets}). This implies that
$\closure{\prod_{i\in I}Y_i}\subset\prod_{i\in I}\closure{Y_i}$, since $\closure{\prod_{i\in I}Y_i}$ is the smallest closed set containing $\prod_{i\in I}Y_i$, and $\prod_{i\in I}Y_i\subset\prod_{i\in I}\closure{Y_i}$ which is closed, so $\prod_{i\in I}Y_i\subset \closure{\prod_{i\in I} Y_i}\subset \prod_{i\in I}\closure{Y_i}$.
\item[$\boxed{\supset}$] Conversely, let $y\in \prod_{i\in I}\closure{Y}_i$, we want to prove that it is also in the closure $\closure{Y}=\closure{\prod_{i\in I} Y_i}$.
For this we are going to use the recently proved result of \lemref{lem:interior-closure-points}, i.e. we are going to prove that every neighbourhood of $y$ has a non-empty intersection with $\prod_{i\in I}Y_i$. We can take a base neighbourhood $U=\prod_{i\in I} A_i$, such that $\exists  I_0\subset I$:
\[
	A_i = \begin{cases}
		B_i \in \B_{i} & \text{if } i\in I_0\\
		X_i & \text{if } i\in I\setminus I_0
	\end{cases}
\]
with $\B_{i}$ being a base for $X_i$. For each $i\in I_0$ we have that $\pi_{i}(y)=y_i\in (A_i\cap \closure{Y_i})$ (because $U$ is neighbourhood of $y$). We need to prove that, there is a $y_i'\in (A\cap Y_i)$ which would mean that $(\prod_{i\in I}A_{i}) \cap (\prod_{i\in I}Y_i)\neq \emptyset$.
We know that:
\[
  y\in \left(\prod_{i\in I} \closure{Y_i}\right)\cap \left(\prod_{i\in I_0}B_i\times \prod_{i\in I\setminus I_0} X_i\right)
\]
Let us analyse the two cases:\\
\\
\noindent{}$\boxed{i\in I_0}$ $y_i\in (B_i\cap \closure{Y_i})$ $\Then$ $y_i\in\setdef{x_i\in X_i}{(x_i\in B_i)\land (x_i\in \closure{Y_i})}$ $\Then$ $\exists x_i\in X_i$ such that $x_i\in B_i$ and $x_i\in Y_i$ (because $Y_i\subset \closure{Y}_i$).\\
\\
\noindent{}$\boxed{i\in I\setminus I_0}$ $y_i\in (X_i\cap \closure{Y_i}) =\closure{Y_i}$ $\Then$ $\exists z_i\in Y_i=(X_i\cap Y_i)$ (indeed, it could be the very same $y_i$)\\
\\
Thus, we can define $y'\in X$ as follows:
\[
	y' = \begin{cases}
		x_i & \text{if } i\in I_0\\
		z_i & \text{if } i\in I\setminus I_0.
	\end{cases}
\]
This proves that $y'\in (U\cap\prod_{i\in I} Y_i)$ $\Then$ $U\cap \prod_{i\in I} Y_i\neq \emptyset$ $\overset{\lemref{lem:interior-closure-points}}{\Then}$ $y\in \closure{\prod_{i\in I} Y_i}$.
\end{itemize}
Finally, to prove the particular case, we only need to use the definition of complement of a set, and the fact that $\mathrm{cl}^\complement(Y_i)$ is open in $X_i$. Thus, let $Y=\prod_{i\in I}\closure{Y_i}$:
	\[
		Y^{\complement}=\left(\prod_{i\in I}X_i\right)\setminus\left(\prod_{i\in I}\closure{Y_i}\right)=\prod_{i\in I}(X_i\setminus\closure{Y_i})=\prod_{i\in I}\mathrm{cl}^\complement(Y_i)=\bigcap_{i\in I} \pi^{-1}_i[\mathrm{cl}^\complement(Y_i)]
	\]
Given that $\mathrm{cl}^\complement(Y_i) \in \T_{i}$, it can be expressed as the union of elements of the base $\B_{i}$. We can assume that $\mathrm{cl}^\complement(Y_i) = B_i\in\B_{i}$ without loss of generality.
Then, it is easy to see that: $Y^\complement = \bigcap_{i\in I} \pi_i^{-1}[B_i]$ which is open in $X$ because it is the intersection of elements of a subbase of the product topology (see~\defref{def:product-topology}).
\end{proof}

\begin{remark}
	We might be tempted to think that a similar statement can be made for the interior of the product of subsets.
	Indeed, a finite product of interior subsets equals the interior of the product:
	\[
		\Interior{\prod_{i=1}^n Y_i} = \prod_{i=1}^n \interior{Y_i}.
	\]
	In particular, the product of interior sets is open.
	However, this cannot be generalised to infinite products of subsets $Y_i\subsetneq X_i$.
	This is because the open sets in the product topology are of the form:
	\[
		\T_X \ni U = \prod_{i\in I_0} U_i \times \prod_{i\in I\setminus I_0} X_i,\ I_0\subset I.
	\]
	It is easy to check that these sets are not contained in the interior of the product of sets, because they contain points outside the product of the interiors of the sets, or to make it more explicit:
	\[
		\prod_{i\in I_0} B_i \times \prod_{i\in I\setminus I_0} X_i \not\subset \prod_{i\in I_0} \interior{Y_i} \times \prod_{i\in I\setminus I_0} \interior{Y_i}.
	\]
\end{remark}

\begin{definition}
	\label{def:dense-set}
	Consider a topological space $(X, \T)$, and a subset $A\subset X$. We say that $A$ is \textbf{dense} in $X$, or that $A$ is a \textbf{dense subset} of $X$, if any of the following equivalent conditions hold:
	\begin{itemize}
		\item The closure of $A$ is the whole space, i.e. $\closure{A}=X$.
		\item The interior of the complement of $A$ is empty, i.e. $\interior{A^\complement}=\emptyset$.
		\item $A$ intersects every non-empty open subset of $X$, i.e. $\forall U\in\T\setminus\{\emptyset\}$, $U\cap A\neq\emptyset$.
	\end{itemize}
\end{definition}

\begin{remark}
	In other words, a subset is \emph{dense} in the topological space $(X,\T)$ if every point in $X$ is either in $A$ or is arbitrarily close to a point in $A$.
\end{remark}

In a similar way as closed sets were introduced as the dual concept of open sets, the dual concept of dense sets is the so-called \emph{nowhere dense} (or \emph{rare}) sets:

\begin{definition}
	\label{def:nowhere-dense-set}
	Consider a topological space $(X, \T)$, and a subset $A\subset X$. We say that $A$ is \textbf{nowhere dense} in $X$, or that $A$ is \textbf{rare} in $X$, if the complement of its closure is dense in $X$. By using~\defref{def:dense-set}, this means the following equivalent conditions hold:
	\begin{itemize}
		\item The complement of $\closure{A}$ is the whole space, i.e. $(X\setminus \closure{A}) = X$.
		\item The interior of the closure of $A$ is empty, i.e. $\interior{\closure{A}}=\emptyset$.
		\item There exists a non-empty open subset of $X$ disjoint from $A$, i.e. $\exists V\in\T\setminus\{\emptyset\}$, such that $V\cap A=\emptyset$.
	\end{itemize}
\end{definition}

It is straightforward to check that~\defref{def:nowhere-dense-set} is the dual concept (i.e., negation)~\defref{def:dense-set}.

\begin{example}
	Both rational $\mathbb{Q}$ and irrational $\mathbb{R}\setminus\mathbb{Q}$ numbers are dense in $\mathbb{R}$ (with the usual topology). This is because every open set that is not empty contains an open interval (also not empty), and every open interval in $\mathbb{R}$ contains both rational and irrational numbers. If we want to rephrase this in terms of \emph{proximity}, we can see that every real number $x\in \mathbb{R}$ is either in $\mathbb{Q}$ or is arbitrarily close to a rational number, which can be found via the famous \emph{Diophantine approximation}, named after Diophantus of Alexandria. And, the same holds for irrational numbers, which is achieved via \emph{continued fractions}, and the resultant approximation is referred to as the \emph{continued fraction representation} of the number.
\end{example}

The following result establishes an interesting property of dense sets which will be useful for later results:

\begin{lemma}
	Let $(X,\T)$ be a topological space, $D\subset X$ a dense subset of $X$, and $U\subset X$ an open subset. Then, we have that $\closure{U}=\closure{U\cap D}$.
\end{lemma}
\begin{proof}
Let us prove the inclusion in both directions:
\begin{itemize}
	\item[$\boxed{\subset}$] Let $x\in \closure{U}$, and assume $x\not\in \closure{U\cap D}$ $\Then$ $x\in X\setminus (\closure{U\cap D})$ $\Then$ $x\in \Interior{X\setminus(U\cap D)} = \interior{U^\complement}\cup \cancelto{\emptyset}{\interior{X\setminus D}}$ $\Then$ $x\in \interior{U^\complement}$. By using \thref{th:interior-closure-complement-relations} we know that $\interior{U^\complement}=X\setminus \closure{U}$ $\Then$ $x\in \closure{U}$ and $x\in (X\setminus \closure{U})$, which is a contradiction. This proves that $x\in \closure{U\cap D}$.
	\item[$\boxed{\supset}$] This proof is trivial, since $(U\cap D)\subset U$ by definition $\overset{\lemref{lem:interior-closure-subsets}}{\Then}$ $(\closure{U\cap D})\subset \closure{U}$, which proves exactly what we wanted.
\end{itemize}
\end{proof}

Before we finish this section, it is worth anticipating that the concept of \emph{dense sets} will take an alternative, though related, interpretation within the context of ordered spaces.
The new meaning will give rise to the concept of \emph{dense order}, which intuitively means that there are no ``gaps'' in the order, i.e. between any two elements there is another element.

\subsubsection{Regular spaces}\label{subsubsec:regular-spaces}

In what follows we are going to introduce another type (i.e., another characterisation) of topological spaces yet more restrictive than the Hausdorff spaces.
Although it could seem that this is the next step in the hierarchy of topological spaces (or, or separation axioms), there is an intermediate property usually denoted as $T_{2\,\text{\textonehalf}}$, or \emph{Urysohn}'s property.
Urysohn's property differ from $T_2$ in that it requires the separating (i.e., disjoint) neighbourhoods to be closed, which is a stronger condition than disjoint open neighbourhoods.
With this in mind, we can now introduce the definition of \emph{regular spaces} as topological spaces in which closed sets and points can be separated by open sets.

\begin{definition}
	\label{def:regular-space}
	A topological space $(X,\T)$ is said to be \textbf{regular}, or that is type $T_3$, if it fulfills the following condition (separation axiom):
	\begin{itemize}
		\item[$T_3$:] $X$ is a $T_1$ space, and for every closed set $F\subset X$ and every point $p\in X\setminus F$, there exist disjoint open sets $U,V\in\T$ such that $p\in U$ and $F\subset V$.
	\end{itemize}
\end{definition}

\begin{remark}
	By definition, we have required that regular spaces are $T_1$.
	As we know, this which means that every singleton is closed in $X$.
	This allows us to easily show that every $T_3$ space is also Hausdorff ($T_2$), because for every $x\neq y$ we know that $x\notin \{y\}$ $\Then$ $x\in X\setminus\{y\}$, which is open because $\{y\}$ is closed.
	Therefore, given that $X$ is $T_3$, $\exists U,V\in\T$ such that $x\in U$ and $\{y\}\subset V$ $\Then$ $x\in U$ and $y\in V$ and $U\cap V=\emptyset$, thus $X$ is $T_2$. The opposite is not true in general, i.e. Hausdorff spaces are not necessarily regular,
	as we will see in the following example.
\end{remark}

\begin{example}
	Consider $\mathbb{R}$ with the topology induced by the base:
	$\B=\setdef{B_i(x)}{i\in\mathbb{N},\,x\in\mathbb{R}}$ with:
	\[
		B_i^*(x)=\begin{cases}
					(x-\frac{1}{i},x+\frac{1}{i}) & \text{if } x\neq 0\\
					(-\frac{1}{i},\frac{1}{i})\setminus\setdef{\frac{1}{m}}{m\in\mathbb{Z}\setminus\{0\}} & \text{if } x=0.
		\end{cases}
	\]
	This is a slight modification of the so-called \emph{Smirnov's deleted sequence topology}, also called $K$-topology after introducing the set $K=\{\frac{1}{m}\,:\,m\in\mathbb{Z}\setminus\{0\}\}$.
	The reader can check that this is a topology, and that it is Hausdorff.
	The fact that this is not a $T_3$ space can be checked by observing that the set $K$ is closed in the topology, but cannot be separated from the point $x=0$ by disjoint open sets.
	Let us take $p=0\in X$, and $K$ as closed set, and take two open sets $U,V$ such that $p\in U$ and $K\subset V$.
	Now, $p\in U$ $\Then$ $\exists i\in\mathbb{N}$ such that $p\in B_i^*(0)\subset U$.
	At the same time $y=\frac{1}{i+1}\in K\subset V$ $\Then$ $\exists j\in\mathbb{N}$ such that $B_j^*(y)\subset V$.
	It is clear that $B_i^*(0)\cap B_j^*(y)\neq\emptyset$
	Then $U\cap V\neq \emptyset$, so $X$ is not regular.
\end{example}

The following two results are two useful characterisations of regular spaces:

\begin{theorem}
	\label{th:regular-space-characterisation-1}
	Let $(X, \T)$ be a $T_1$ space, then it is regular \iff\ $\forall x\in X$ and every $V\subset X$ neighbourhood of $x$, $\exists U\subset X$ also neighbourhood of $x$ such that $x\in U\subset \closure{U}\subset V$. In fact, this also holds true for any neighbourhood $V$ belonging to a subbase of the topology.
\end{theorem}

\begin{proof}
	Let us proceed with the proof of both implications:
	\begin{itemize}
		\item[$\boxed{\Rightarrow}$] Since $X$ is $T_3$, we know that for any $x\in X$ and any closed set, e.g. $X\setminus \interior{V}$, $\exists U,W\in \T$ so that $x\in U$ and $(X\setminus \interior{V})\subset W$, with $U\cap W=\emptyset$. Then, $U\subset W^\complement$ with $W^\complement$ being a closed set $\Then$ $\closure{U}\subseteq\closure{W^\complement}$. Givent that $X\setminus \interior{V}\subset W$ we have that $W^\complement \subset X\setminus (X\setminus \interior{V}) = \interior{V}\subset V$. Therefore, $U\subset \closure{U}\subset \closure{W^\complement}\subset \interior{V}\subset V$, which proves the implication.
		\item[$\boxed{\Leftarrow}$] We are going to prove it first without the restriction of $V$ being a subbase neighbourhood ($^\clubsuit$), and later with the restriction ($^\blacklozenge$).\\
		\\
		$^\clubsuit$ Let us take $x	\in U\subset\closure{U}\subset V$, with $U,V$ being neighbourhoods of $x\in X$. We can confirm that $x\in\interior{V}$, and $X\setminus \interior{V}$ is closed. Given that $V$ is any neighbourhood, $X\setminus \interior{V}$ is any closed set. Since $U$ is neighbourhood of $x$, $\exists W\in \T$ such that $x\in W\subset U$.
		We know that $U\cap U^\complement = \emptyset$, and $U\subset \closure{U}$, thus $U\cap (X\setminus \closure{U})=\emptyset$ $\Then$ $X\setminus \closure{U}\subset U^\complement$.
		Now, $W\subset U\subset \closure{U}\subset V$ $\Then$ $(X\setminus\interior{V})\subset (X\setminus\closure{U})=Z$, which is open. Therefore, $\exists W, Z\in \T$ such that $x\in W$, $(X\setminus \interior{V})\subset Z$, and $W\cap Z = \emptyset$. This proves that $X$ is $T_3$.\\
		\\$^\blacklozenge$ Let us restrict now to subbase neighbourhoods. Take $x\in X$ and $F\subset X$ closed, such that $x\in F^\complement$, hence $F^\complement$ is an open neighbourhood of $x$. Therefore, $\exists V_1,\dots, V_n$ subbase elements such that $x\in\bigcap_{i}V_i\subset F^\complement$. By hypothesis, $\exists U$ neighbourhood of $x$ such that $U\subset \closure{U} \subset \bigcap_{i}V_i$, or what is equivalent: $\exists U_1,\dots, U_n$ subbase elements such that $x\in U_i\subset \closure{U_i}\subset V_i$. If we take $U=\bigcap_{i}U_i$ and $V=X\setminus\bigcap_{i}\closure{U}_i$ which are disjoint sets $\Then$ $x\in U$ and $F\subset (X\setminus \bigcap_{i=1}^n V_i) \subset (X\setminus \bigcap_{i=1}^n\closure{U}_i)=V$. This proves that $X$ is $T_3$.
	\end{itemize}
\end{proof}

\begin{theorem}
	\label{th:regular-space-characterisation-2}
	A topological space $(X, \T)$ is regular \iff\ it is $T_1$, and $\forall x\in X$ there is a closed neighbourhoods base for $x$.
\end{theorem}

\begin{proof}
Let us proceed with the proof of both implications:
	\begin{itemize}
		\item[$\boxed{\Rightarrow}$] This follows directly from \thref{th:regular-space-characterisation-1}, because if $X$ is $T_3$ then $\forall x\in X$ and every neighbourhood $V$ of $x$, $\exists U$ also neighbourhood of $x$ such that $x\in U\subset \closure{U}\subset V$. Then, we can take $C=\closure{U}$, which is closed, and $C\subset V$, which proves that $C$ is a closed neighbourhood of $x$.
		\item[$\boxed{\Leftarrow}$] Let $X$ be $T_1$, $x\in X$, and $F\subset X$ closed, such that $x\in F^\complement$. By hypothesis, there exists a base of closed neighbourhoods for $x$ $\Then$ $\exists C\subset X$ closed such that $x\in C\subset F^\complement$.
		We can take $U=\interior{C}$ and $V=C^\complement$ $\Then$ $x\in U$ (because $X$ is $T_1$), $F\subset V$ (because $C\subset F^\complement$), and $U\cap V =\emptyset$. Therefore $X$ is $T_3$.
	\end{itemize}
\end{proof}

In the following we are going to show that the regularity property is preserved when taking subspaces, and that the product of regular spaces is also regular.

\begin{theorem}
	\label{th:regular-subspace}
	Every subspace $Y\subset X$ (i.e, $Y$ equipped with the subspace topology) of a regular space $X$ is also regular.
\end{theorem}

\begin{proof}
	First, we need to prove that $Y$ is $T_1$, which is the first requirement of the regularity axiom (see~\defref{def:regular-space}).
	This is straightforward to prove, because $Y\subset X$ and $X$ is $T_1$ by definition. Thus, we have that $Y$ is $T_1$ according to~\thref{th:subspace-separation-consistency}.

	Let $A\subset Y$ be a closed subset in $Y$, and let $x\notin A$, i.e. $x\in A^\complement$. Since $A$ is closed in $Y$, $\exists F\subset X$ such that $A=Y\cap F$, with $F$ closed in $X$ (see~\lemref{lem:closed-sets-subspace-topology}). Since $x\notin A$, it is clear that $x\notin F$ $\Then$ $x\in X\setminus$. Then, we have $x\in F^\complement$ and $F$ closed set in $X$, and we know that $X$ is regular $\Then$ $\exists V, W\in \T$ ($V\cap W =\emptyset$) with $x\in V$, and $F\subseteq W$. Thus, the following holds:
	\[
		x\in\underbrace{(Y\cap V)}_{\in \Trel{Y}}\land \underbrace{(Y\cap F)}_{\text{closed in }Y}\subseteq \underbrace{(Y\cap W)}_{\in \Trel{Y}}
	\]
	This means that for every $A$ closed in $Y$, and every $x\in (Y\setminus A)$, $\exists P,Q\in\Trel{Y}$ such that $x\in P$ and $A\subset Q$, with $P\cap Q=\emptyset$. Therefore, $Y$ is regular.
\end{proof}

\begin{theorem}
	\label{th:regular-product-space}
	Let $\{X_i\}_{i\in I}$ be a family of non-empty spaces.
	Then, the cartesian product $X=\prod_{i\in I}X_i$ equipped with the product topology is regular \iff\ every $X_i$ is regular.
\end{theorem}

\begin{proof}
	Let us proceed with the proof of both implications:
	\begin{itemize}
		\item[$\boxed{\Rightarrow}$] Let $X=\prod_{i\in I}$ be a regular space, and let us focus our attention on a particular factor space $X_j$. Let $x\in X_j$ and $F\subset X_j$ closed in $X_j$, such that $x\in (X_j\setminus F)$. Given that $X$ is regular, we know that for every $x\in X$ and every closed set $C\subset X$ such that $x\in (X\setminus C)$, $\exists U,V\in\T$, with $U=\prod_{i\in I}U_i$ and $V=\prod_{i\in I}V_i$, such that $x\in U$ and $C\subset V$ with $U\cap V=\emptyset$. We can take $x$ such that $\pi_j(x)=x_j$, and $C$ so that $\pi_j[C]=F$. Then, we have:
		\[
			\pi_j[x]\in \pi_j[U]\ \land\ \pi_j[C]\subset \pi_j[V]\ \land\ \pi_j[U]\cap \pi_j[V]=\emptyset
		\]
		Therefore, $\exists U_j,V_j\in\T_j$ such that $x_j\in U_j$ and $F\subset V_j$ with $U_j\cap V_j=\emptyset$. This proves that $X_j$ is regular.
		\item[$\boxed{\Leftarrow}$] By hypothesis, $X_i$ is regular $\forall i\in I$ $\Then$ $X_i$ is $T_1$, $\forall i\in I$ $\overset{\text{\thref{th:subspace-separation-consistency}}}{\Then}$ This proves that $\prod_{i\in I}X_i$ is also $T_1$.\\
		Now, we need to prove that the product space is also regular, and for this we will use the result~\thref{th:regular-space-characterisation-1} with the subbase formed by the subsets $\pi_i^{-1}[V_i]$, where $V_i\in\T_i$. Let $x\in\pi_i^{-1}[V_i]\in\T_X$, so that $x_i\in V_i$. Since every $X_i$ is regular, $\exists U_i\in \T_i$ such that $x_i\in U_i\subset\closure{U_i}\subset V_i$ $\Then$ $x\in \pi_i^{-1}[U_i]\subset \closure{\pi_i^{-1}[U_i]}=\pi_i^{-1}[\closure{U_i}]\subset \pi_i^{-1}[V_i]$, where we have used ~\defref{def:projection} and~\thref{th:closure-product-space} for the identity $\closure{\pi_i^{-1}[U_i]}=\pi_i^{-1}[\closure{U_i}]$. Thus, $\forall x\in X$ and every $V=\pi^{-1}_i[V_i]\subset X$ subbase neighbourhood of $x$, $\exists U\subset X$, also neighbourhood of $x$ such that $x\in U\subset\closure{U}\subset V$ $\Then$ $X$ is regular.
	\end{itemize}
\end{proof}

\subsubsection{Boundary points}\label{subsubsec:boundary-points}

There are several equivalent definitions of the boundary of a subset $S\subset X$ of a topological space $(X,\T)$, which we will refer to as $\partial_{X}S$ or $\mathrm{Bd}_{X}S$, or simply $\partial S$ when the space can be inferred from the context. We can proceed with the introduction of the first definition, to later introduce its equivalent formulations.

\begin{definition}
	\label{def:boundary-set}
	Let $(X, \T)$ be a topological space, and $A\subset X$ a subset. We define the \textbf{boundary} of $A$ as follows:
	\[
		\partial A = \closure{A}\cap\closure{(X\setminus A)}
	\]
\end{definition}

The boundary of a set $A$ is constituted by points which are in the closure of $A$, but also in the closure of the complement of $A$.
In other words, the points of the boundary of $A$ have as neighbours points both in $A$ and in $X\setminus A$, therefore they are in the ``boundary'' between $A$ and its complement.
By straightforward application of~\thref{th:interior-closure-complement-relations}, we can go from this to the other equivalent definitions:
\[
	\partial A = \closure{A}\cap\closure{(X\setminus A)}=\closure{A}\cap(X\setminus\interior{A})=\closure{A}\setminus\interior{A}
\]

Thus, the boundary of a set comprises the points which demarcate the difference between the interior and the closure of the set (which is exactly what the last definition states).

\begin{lemma}
	Let $(X,\T)$ be a topological space, and $A\subset X$. Then, $A$ is open in $X$ \iff\ $\forall x\in \partial A$, $x\notin A$. Or, equivalently, $A$ is closed in $X$ \iff\ $\forall x\in \partial A$, $x\in A$.
\end{lemma}

\begin{proof}
	Let us prove both implications:
	\begin{itemize}
		\item[$\boxed{\Rightarrow}$] Since $A$ is open in $X$, then $A=\interior{A}$. By definition, $\partial A =\{x\in X\,:\,(x\in \closure{A})\land (x\notin \interior{A})\}$ $\Then$ $\forall x\in \partial A$, $x\notin A$.
		\item[$\boxed{\Leftarrow}$] Let $x\in \partial A$, and $x\notin A$ $\Then$ $x\in \closure{A}\setminus{\interior{A}}$ and $x\notin A$. If $A$ is not open, then $A\subset \closure{A}$. By hypothesis, $x\notin A\subset \closure{A}$ $\Then$ $x\notin \closure{A}$, which is a contradiction with $x\in \closure{A}\setminus{\interior{A}}$. Therefore, $A$ is open.
	\end{itemize}
\end{proof}

The previous result leads us to the conclusion that the open subsets in $X$ are those which do not contain any of their boundary points, and the closed subsets are those which contain all their boundary points. The following result shows us that the boundary of a set cannot be ``too big''.

\begin{theorem}
	\label{th:interior-boundary-empty}
	Let $A$ be an open subset in the topological space $(X, \T)$, then $\partial A$ is a closed subset  with empty interior, i.e. $\interior{\partial A}=\emptyset$.
\end{theorem}

\begin{proof}
	The proof is straightforward by using the previous definitions, and using twice~\thref{th:interior-closure-complement-relations}:
	\[
		\Interior{\partial A}=\Interior{\closure{A}\cap(X\setminus \interior{A})}=\interior{\closure{A}}\cap\Interior{(X\setminus \interior{A})}=\interior{\closure{A}}\cap\Interior{(X\setminus A)}=\interior{\closure{A}}\cap(X\setminus \closure{A})=\emptyset
	\]
\end{proof}

\subsubsection{Accumulation points}\label{subsubsec:accumulation-points}

The relation $A\subset \closure{A}$ can be intuitively interpreted as asserting that for every point in $A$ there is another point in $A$ arbitrarily close to it, e.g. $x$ itself. However, it is useful to know whether a point $x\in A$ has surrounding points in $A$ which are not $x$ itself. This is the idea behind the concept of \emph{accumulation point}, which we are going to introduce in this section.

\begin{definition}
	\label{def:accumulation-point}
	Let $A$ be a subset of a topological space $(X,\T)$, and consider $x\in X$. We say that $x$ is an \textbf{accumulation} (or \textbf{limit}) point of $A$ if every neighbourhood of $x$ contains at least one point of $A$ different from $x$ itself.
\end{definition}