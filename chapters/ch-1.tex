\chapter{Topological spaces}\label{ch:topological-spaces}

In this first chapter, we are going to study a fundamental concept which will pervade
the rest of this course, the concept of \emph{topological space}.
Once this concept is introduced, we will proceed with some basic topological concepts,
besides studying the most important operations which allow for the construction of new
topological spaces.

\section{Topologies, bases and subbases}
\label{sec:topologies-bases-and-subbases}

\subsection{Topologies}
\label{subsec:topologies}

The modern definition of topological space is based upon the fundamental concept of
open  set (which is axiomatically introduced).
Thus, a topological space is defined by specifying the family of open sets which form it.
The formal definition is as follows.

\begin{definition}
	\label{def:topology}
	Let $X$ be a set, and let $\T$ be a family of subsets of $X$ (i.e., $\T\subset\P X$).
	Then $\T$ is called a \textbf{topology} on $X$ if the following properties are
	fulfilled:
	\begin{enumerate}
		\item $\emptyset, X\in \T$
		\item Let $\{A_i\}_{i\in I}$ be a family of subsets such that $A_i \in \T$,
		then $\bigcup\limits_{i\in I} A_i\in\T$
		\item Let $A, B\in \T$, then $(A\cap B)\in \T$
	\end{enumerate}
\end{definition}

Having the formal definition for \emph{topology}, it readily follows that a
\textbf{topological space} is a tuple $(X, \T)$, with $X$ being a set and $\T$ a
topology on $X$.
The elements of $\T$ are called the \textbf{open subsets} of $X$.

\begin{remark}
	The property (\defref{def:topology}-2) could be rewritten in a more concise way by
	saying that, if $\F\subset\T$, then $\bigcup\F\in\T$.
	Also, the third property can be readily extended to any finite amount of open sets
	by induction, so that $\forall \{A_k\}_{k=1,\dots,n}$ with $A_k\in\T$, then
	$\bigcap\limits_{k=1}^n A_k\in \T$.
\end{remark}

For practical purposes, we might simply use $X$ instead of $(X, \T)$, keeping in mind
that $\T$ changes with $X$.
The first topological concept we introduce based upon the concept of \emph{open subset} is
the concept of \emph{neighbourhood}.

\begin{definition}
	\label{def:neighbourhood}
	Let $C, U\subset X$, with $(X, \T)$ being a topological space.
	Then, $U$ is said to
	be a \textbf{neighbourhood} of $C$ if $\exists A\in\T$ such that $C\subset A \subset U$.
	The subset $U$ is said to be a neighbourhood of a point $x\in X$ if $U$ is a
	neighbourhood of $\{x\}$.
	An open neighbourhood of $x$ is simply an open subset of $X$ containing $x$.
\end{definition}

The neighbourhoods of the points of a topological space determine the open sets of the topology:

\begin{theorem}
	Let $(X, \T)$ be a topological space.
	A subset $U\subset X$ is an open subset \emph{if and only if} (iff) $U$ is neighbourhood of all its points.
\end{theorem}

\begin{proof} To prove this theorem will proceed with the demonstration of the left and
right implies.

\noindent{$\boxed{\Rightarrow}$} This is trivial, given that $U\in\T$ exists and
fulfils the conditions to be a neighbourhood of all its points.

\noindent{$\boxed{\Leftarrow}$} Let $U$ be a neighbourhood of all its points.
Then, $\forall x\in U$, $\exists V\in \T\,:\,\{x\}\subset V\subset U$ $\Rightarrow$
$U = \bigcup\F$ with $\F=\{V\in\T\,:\,V\subset U\}\,\Then\,U$ is open, since the union
of open subsets is open according to \defref{def:topology}-2.
\end{proof}

The intuitive idea we must have developed so far is that a neighbourhood of a point $x$ is a set
$U$ which contains all the points ``\emph{around}'' $x$, which also means that $U$ is open
\iff\,it only contains all the points surrounding every one of its points.

\begin{example}
	Let $X$ be a set, the \emph{trivial topology} is $\T=\{\emptyset, X\}$, and the
	topology $\T=\P X$ is called the $\emph{discrete topology}$.
\end{example}

The proof that both examples fulfil the conditions required by the definition
\defref{def:topology} is straightforward, and left to the reader if interested.
On the one hand, the trivial topology is the smallest one possible, since it has the
minimal elements required to fulfil the definition \defref{def:topology}.
On the other hand, the discrete topology is the largest one possible, given that all
the possible subsets of $X$ are open.

\begin{definition}
	The elements $x\in X$ will be called \emph{points} (or \emph{singletons}).
	Thus, a point $x\in X$ is said to be \emph{isolated} if $\{x\}$ is an open subset
	of $X$, i.e. $\{x\}\in \T$.
\end{definition}

\begin{remark}
	A trivial result is that a topological space $(X,\T)$ is \emph{discrete} (i.e.,
	$\T=\P X$) \iff\ $\forall x\in X$, $x$ is isolated.
\end{remark}

\subsection{Bases}\label{subsec:bases}

\begin{definition}
	Let $(X, \T)$ be a topological space, and $\B = \{B \subset X\,:\, B \in \T\}\subset\T$
	a family of open sets.
	The family $\B$ is a \textbf{base} for $X$ \iff\, every open subset of $X$ can be
	expressed as the union of elements of $\B$.
	This is equivalent to say that, $\B\subset\T$ is a base for $(X,\T)$ if for every $U\in\T$ and $u\in U$, there is a $B\in \B$ such that
	$u\in B\subset U$.
	\label{def:base}
\end{definition}

The convenience of introducing this concept becomes clear with the following two theorems:

\begin{theorem}
	Let $(X, \T)$ be a topological space, and $\B$ a base for $(X, \T)$.
	The following two properties are satisfied:
	\begin{enumerate}
		\item $\bigcup\limits_{B\in\B} B \equiv \bigcup\B = X$
		\item Let $A, B\in \B$ and $x\in A\cap B$, then $\exists C\in \B$ such
		that $x\in C\subset (A\cap B)$
	\end{enumerate}
\end{theorem}

\begin{proof}
	This proof is straightforward by using \defref{def:topology} and \defref{def:base}:
	\begin{enumerate}
		\item \noindent{$\boxed{\subseteq}$}  $\bigcup\B \subset X$ is trivial.

		\noindent{$\boxed{\supseteq}$} Given that $X\in \T$ (hence it is an open
		subset) $\Then$ $X$ can be expressed as the union of a family $\F\subset \B$
		$\Then$ $X \subset \bigcup \F\subset \bigcup \B$.

		\item Let $A, B\in \B$ $\overset{\text{\defref{def:topology}}}{\Then}$
		$(A\cap B)\in \T$ $\overset{\text{\defref{def:base}}}{\Then}$
		$\exists C\in \B$ such that $x\in C\subset(A\cap B)$.
	\end{enumerate}
	With this, the theorem is proved to hold true.
\end{proof}

Interestingly, the two properties of the previous theorem are sufficient for a set
family to be a base for a topology:

\begin{theorem}
	Let $X$ be a set and $\B\subset \P X$ a family of subsets of $X$ such that:
	\begin{enumerate}
		\item $\bigcup \B = X$
		\item If $A, B\in \B$ and $x\in (A\cap B)$, there exists $C\in \B$ such that
		$x\in C\subset (A\cap B)$
	\end{enumerate}
	Therefore, there exists a unique topology $\T$ on $X$ with $\B$ being a base for it,
	and such topology is the smallest (w.r.t.\footnote{with respect to} the inclusion)
	for which the elements of $\B$ are open.
	\label{th:1-3}
\end{theorem}

\begin{proof}
	To proceed with the proof, we are going to first prove that there exists a topology $\T$, and later
	we will prove that $\T$ is unique.\\
	Let us define $\T = \{\bigcup \F\,:\,\F\subset \B\}$, and now proceed with the proof
	that $\T$ is a topology.
	\begin{enumerate}
		\item By condition (1), $X=\bigcup \B$ $\Then$ $X\in \T$.
		And, it is trivial that $\emptyset = \bigcup \emptyset \in \T$.

		\item Let $\F\subset \T$, and let $\F_A=\{B\in \B\,:\, B\subset A\}$ for each
		$A\in\F$, so that $A=\bigcup\F_A$.
		Let $\F^*=\bigcup\limits_{\F_A\in\F}\F_A\subset\B$.
		Then,  $\bigcup \F =\bigcup\F^*\in\T$.

		\item Let $A, B\in \T$, and $\F = \{C\in\B\,:\, C \subset(A\cap B)\}$.
		Let us prove that $A\cap B = \bigcup\F$:
		\begin{itemize}
			\item[\boxed{\supseteq}] This is a trivial proof: $\forall x\in\bigcup\F$ $\Then$
			$x\in\setdef{C\in\B}{C\subset(A\cap B)}\ \Then\ x\in (A\cap B)$ $\Then$
			$\bigcup\F \subset (A\cap B)$.
			\item[\boxed{\subseteq}] Let $x\in (A\cap B)$, according to $(\emph{2.})$,
			$\exists C\in \B$ such that $x\in C\subset (A\cap B)$ $\Then$ $x\in C\subset
			\bigcup\F$ $\Then$ $(A\cap B)\subset \bigcup\F$.
		\end{itemize}
		Finally, $\bigcup \F\in \T$ $\Then$ $(A\cap B) \in \T$.
	\end{enumerate}
	Therefore, $\T$ is a topology on $X$.
	To prove the final part of the theorem, consider $\T'$ a topology on $X$ with
	$\B$ as base.
	By definition, $\T\subset \T'$.
	Therefore, $\T$ is unique and minimal.
\end{proof}

Thus, to define a topology on a set $X$ one only needs to determine a family $\B$ with
the properties just described.
In what follows, we introduce a related concept which
will be really convenient later on this book, the one of \emph{neighbourhoods base} or
\emph{neighbourhoods system}.

\begin{definition}
	\label{def:neighbourhoods-base}
	Let $(X, \T)$ be a topological space, and $x\in X$.
	An \textbf{(open) neighbourhoods base} for $x$ is a family $\B\subset\T$ of open
	neighbourhoods of $x$ such that, for every $U$ neighbourhood of $x$, $\exists B\in\B$ such that
	$x\in B\subset U$.
\end{definition}

It is clear that if $(X,\T)$ is a topological space, and we have a family
$\{\B_x\}_{x\in X}$ so that $\B_x$ is a neighbourhoods base for $x$, then
$\B=\bigcup\limits_{x\in X}\B_x$ is a base for (X, $\T$).
Reciprocally, let $\B$ be a base for $(X, \T)$, then $\B_x = \setdef{A\in\B}{x\in A}$ is
a neighbourhoods base for $x$.

\begin{remark}
	In some references, the concept of \emph{neighbourhoods base} is also called
	\emph{neighbourhoods system}, and it is also usual to make a difference between
	open neighbourhoods bases and neighbourhoods bases, the former usually called
	\emph{local bases}. In this text we will not make such distinction, and we will
	always refer to both as \emph{neighbourhoods base}.
\end{remark}

In this context, it is useful to introduce the concept of \emph{filter}
which has as special case the concept of \emph{neighbourhoods base}:

\begin{definition}
	\label{def:filter}
	Let $(X, \T)$ be a topological space, and $\B$ a family of subsets of $X$ such that:
	\begin{itemize}
		\item $X\in\B$ and $\emptyset\notin\B$
		\item If $A, B\in\B$, then $A\cap B\in\B$
		\item If $A,B\subset X$, $A\in\B$ and $A\subset B$, then $B\in\B$
	\end{itemize}
\end{definition}

It is easy to see that a neighbourhoods base $\B$ for $x\in X$ is a filter, since
$\B$ is a family of open sets, and it is closed under finite intersections.


\subsection{Subbases}\label{subsec:subbases}

In this section we will introduce a concept similar to \emph{base}, which will also be
useful when defining topologies.

\begin{definition}
\label{def:subbase}
	Let $(X, \T)$ be a topological space.
	A family $\S\subset \T$ is a \textbf{subbase}	for $(X, \T)$ \iff\ the family of all finite intersections
	of elements of $\S$, namely	$\I(\S)$, is a base for $(X, \T)$\footnote{The family of all finite intersections of
	elements of $\S$ means: $\I(\S) = \bigcap\limits_{S\in\mathcal{F}} S$, for all finite subsets
	$\mathcal{F}$ of $\S$. In particular for all subsets $\mathcal{F}$ of cardinality
	0 (empty set, $\mathcal{F}_0=\emptyset$) or 1 (singletons, $\mathcal{F}_1=\{A\}$).}.
\end{definition}

The difference between subbases and bases is that the family of subsets does not have
to satisfy any specific condition to be a subbase:

\begin{theorem}
	Let $X$ be a set, and $\S\subset \P X$ a family of subsets of $X$.
	There exists a unique topology on $X$ with $\S$ as subbase, and it is smallest
	(w.r.t.\,the inclusion) for which the elements of $\S$ are open.
	\label{th:1-4}
\end{theorem}

\begin{proof}
	To prove this result we will show that:
	\[
	\B\doteq \I(\S)=\left\{\bigcap\limits_{S\in\mathcal{F}} S=\bigcap\mathcal{F}\,:\,\mathcal{F}\subset \S\right\},
	\]
	satisfies the conditions of \thref{th:1-3}, as follows:
	\begin{enumerate}
		\item $\mathcal{F}_{0}=\emptyset$ is a finite subset, and $X=\bigcap \emptyset\in\B$, therefore $X\subset
		\bigcup\B$.
		Besides, all other finite subsets are $\mathcal{F}_{1}=\{A\}$ with $A\in\S
		\subset\P X$, $\bigcap\mathcal{F}_{1}\subset X$.
		Therefore $\bigcup \B\subset X$.
		Thus $\bigcup\B = X$.
		\item Let $A, B\in \B$, and $x\in (A\cap B)$.
		By definition, $A\in\B \Rightarrow A=\bigcap S$,
		and $B\in \B \Rightarrow B =\bigcap T$, with $S, T\subset \S$.
		Then, $A\cap B = \left(\bigcap S\right)\cap\left(\bigcap T\right)=\bigcap (S\cup T)$.
		It is then clear that $\exists C\in\B$ such that $x\in C\subset (A\cap B)$.
	\end{enumerate}
	Thus, the conditions for \thref{th:1-3} are fulfilled which means that
	$\B$ is base for a topology in $X$, with $\S$ as a subbase.
	Finally, to prove the uniqueness, let us consider a topology $\T'$ on $X$, 
	such that $\forall S\in \S, S\in \T'$ $\Rightarrow$ $\forall B\in\B, B\in\T'$ 
	$\Rightarrow$ $\T\subset \T'$, which also proves that $\B$ is the smallest.
\end{proof}

\begin{example}
	\footnote{This will be discussed in further detail later on the book, when
	we endow any ordered space with a topological structure.}
	The \textbf{order topology} on $\mathbb{R}$ is the topology generated by the
	basis $\B=\setdef{(a,b)}{a,b\in\mathbb{R}, a<b}$. At the same time, we also have
	the usual-metric induced topology on $\mathbb{R}$, which is the topology
	generated by the basis $\B=\setdef{B_\epsilon(x)}{\epsilon>0, x\in\mathbb{R}} = \setdef{(x-\epsilon, x+\epsilon)}{\epsilon>0, x\in\mathbb{R}}$.
	It turns out that both the order topology and the metric topology are the same.
	This is a consequence of the fact that every open of the metric topology is an
	open interval, hence it is also an open set of the order topology. And, conversely,
	if $(a,b)$ is an open on the order topology, and $x\in\mathbb{R}$, then can simply
	take $\epsilon=\min\{x-a, b-x\}$, and we have that $B_\epsilon(x)=(x-\epsilon, x+\epsilon)\subset (a,b)$.
	Therefore, all the open of the order topology are open of the metric topology, and
	vice versa.
\end{example}

\subsection{Product topology}\label{subsec:product-topology}

At this point, it seems natural to try and generalise the concept of topological space to
the cartesian product of topological spaces. The idea is that we have a family of
topological spaces $\{(X_i, \T_i)\}_{i\in I}$ (sometimes we will refer to them simply
as $\{X_i\}_{i\in I}$), and we want to equip the cartesian product
$X=\prod_{i\in I} X_i$ with a topological structure.
Before we proceed with the formal introduction of any topology of the product space,
such as the na\"{i}ve \emph{box topology}, or
the \emph{product topology} (or,
\emph{Tychnoff topology} due to the fact that it was introduced by Andrey Tychnoff in 1935\footnote{%
	Tikhonov, Andrey Nikolayevich (1935). \emph{\"{U}ber einen Funktionraum}, Mathematische Annalen (111): 762--766.
	Cited in Hocking, John G. \& Young, Gail S. (1961). Topology. Addison-Wesley.%
}), we will briefly introduce the concept in the more familiar context of $\mathbb{R}^2$.

\subsubsection{Neighbourhoods in $\mathbb{R}^2$}\label{subsubsec:neighbourhoods-in-r2}
We expect the reader to be familiarised with the concept of the Euclidean (or, simply, \emph{natural}) distance
in $\mathbb{R}^2$, although this will be thoroughly discussed when we deal with metric spaces in Chapter~\ref{ch:metric-spaces}.

The Euclidean distance between two points $x=(x_1,x_2)$ and $y=(y_1,y_2)$ in $\mathbb{R}^2$ is defined as follows:
\begin{equation}
	d(x,y) = \sqrt{(x_1-y_1)^2+(x_2-y_2)^2}.\label{eq:euclidean-distance}
\end{equation}
which represents the length of the segment joining $x$ and $y$.
This distance can be used to define the \emph{open ball} centred at $x$ with radius $\epsilon$ as follows:
\begin{equation}
	B_\epsilon(x) = \{z\in\mathbb{R}^2\,|\,d(x,z)<\epsilon\}.
	\label{eq:open-ball}
\end{equation}
which certainly contains $x$, and all the points \emph{around} $x$ at a distance.
And, given that it contains all the points \emph{around} $x$, it is clear that $B_\epsilon(x)$ is an open set
in $\mathbb{R}^2$.

\begin{remark}
	Using the standard notation $(a,b)\subset\mathbb{R}$ to refer to the
	open interval between $a$ and $b$ would not be convenient, since we are using
	the same notation for coordinates in $\mathbb{R}^2$.
	Whenever this potential misunderstanding is possible, we will use the notation
	$I_{(a,b)} = \setdef{x\in\mathbb{R}}{a<x<b}$.
\end{remark}

However, if our main concern is to determine when the subsets of $\mathbb{R}^2$ which contain the points
\emph{around} $x\in\mathbb{R}^2$, we can either use open balls or, instead, open
rectangles centred at $x$, given that the rectangle:
\begin{align*}
R_\epsilon(x) =& \{z\in\mathbb{R}^2\,|\,|x_1-z_1|<\epsilon_1, |x_2-z_2|<\epsilon_2\} \\
=& I_{(x_1-\epsilon_1, x_1+\epsilon_1)}\times I_{(x_2-\epsilon_2, x_2+\epsilon_2)}
\end{align*}
also contains all the points which are intuitively \emph{around} $x$, hence it is also an
open set in $\mathbb{R}^2$.

These open rectangles generalise to higher dimensions in the obvious way, where the
term \emph{open box} seems more appropriate.
The main advantage of using open boxes,
instead of using open balls, is that they allow us to come up with an easy procedure
to equip the product of two topological spaces $X \times Y$ with a topological structure.
This procedure is based on the observation that the product of open sets
$U \times V$ form a base for the product topology on $X \times Y$.
And, indeed, this observation is true for any finite product of topological spaces.

Unfortunately, this seemingly simple procedure does not work for infinite products of
topological spaces, as was noticed by Tychnoff in 1935.
The problem was that the product of open sets in an infinite product of topological
spaces does not satisfy certain \emph{desirable} properties which we will discuss later.
Tychonoff found a way to overcome this problem, which gave rise to the \emph{product topology}.

\subsubsection{Product topology: Tychonoff's definition}\label{subsubsec:product-topology}

Before being able to introduce the definition for the product topology, we need to
introduce the concept of cartesian \emph{projection}.

\begin{definition}
	\label{def:projection}
	Let $\{X_i\}_{i\in I}$, with $I\subset\mathbb{N}$, be a family of sets, and
	$X=\prod_{i\in I} X_i$ their cartesian product.
	The \textbf{cartesian projection} (or, simply, projection) is the function defined as follows:
	$$
  	\begin{array}{rrcl}
		\pi_j : & X &\longrightarrow & X_j,\\
		 & x &\longmapsto     & \pi_j(x),
  	\end{array}
	$$
	where $\pi_j(x)=x_j \in X_j$ is the $j$-th coordinate of $x$.
	This definition can be extended to any subset $A=\prod_{i\in I}A_i\subset X$
	by defining $\pi_j[A] = \setdef{\pi_j(x)}{x\in A}\linebreak = A_j\subset X_j$.
	The \emph{inverse image mapping} induced by the projection $\pi_j$ is defined as follows:
	$$
		\begin{array}{rrcl}
			\pi^{-1}_j : & X_j &\longrightarrow & X,\\
			 & x_j &\longmapsto     & \pi^{-1}_j(x_j),
		\end{array}
	$$
	where $\pi^{-1}_j(x_j)=\{x\in X\,:\,\pi_j(x)=x_j\} \subset X$.
	This can also be expressed as $\pi^{-1}_j(x_j) = \{x_j\} \times \prod_{i\in I\setminus\{j\}}X_i \subset X$.
	This definition can also be extended to any subset $A_j\subset X_j$ by defining
	$\pi^{-1}_j[A_j]=\setdef{\pi^{-1}_j(x_j)}{x_j\in A_j}=A_j\times\prod_{i\in I\setminus\{j\}}X_i\subset X$.
\end{definition}

From the above definition, we can deduce the following lemma which will be useful later on to prove some results:

\begin{lemma}
	\label{lem:projection}
	Let $\{X_i\}_{i\in I}$ be a family of sets, and $X=\prod_{i\in I} X_i$ their cartesian product.
	Then, the following properties hold true:
	\begin{enumerate}
		\item Let $A_j\subset X_j$, then $(\pi_j\circ\pi^{-1}_j)[A_j] = \text{id}_{A_j}$.
		\item Let $A\subset X$, then $A = \bigcap_{j\in I}(\pi_j^{-1}\circ\pi_j)[A]$.
	\end{enumerate}
\end{lemma}
\begin{proof}These are very simple results, which only require the use of the definitions \defref{def:projection}:
	\begin{enumerate}
		\item Let us use the definition of the inverse first, we have $\pi_{j}^{-1}[A_j]=\{ x\in X\,:\,\pi_{j}(x)=x_j\in A_j\}$.
		Then, $(\pi_{j}\circ\pi_{j}^{-1})[A_j]=\{\pi_{j}(x)\,: x\in \pi_{j}^{-1}[A_j]\}=\{x_j\in A_j\} = A_j$.
		\item {$\boxed{\subseteq}$} Let $x\in A$, then $x\in \prod_{i\in I} A_i$. Given that $\pi_j^{-1}[A_j] = A_j\times\prod_{i\in I\setminus\{j\}}X_i$,
		we have that $x\in \pi_j^{-1}[A_j]$ for each $j\in I$. Therefore, $x\in \bigcap_{j\in I}(\pi_j^{-1}\circ\pi_j)[A]$.\\
		{$\boxed{\supseteq}$} Let $x \in \bigcap_{j\in I}(\pi_j^{-1}\circ\pi_j)[A]$, then $x\in \pi_j^{-1}[A_j]$ for each $j\in I$.
		This means that $x_j \in A_j$ for each $j\in I$, hence $x\in \prod_{i\in I} A_i$.
	\end{enumerate}
\end{proof}

\begin{definition}
	\label{def:product-topology}
	Let $\{(X_i, \T_i)\}_{i\in I}$ be a family of topological spaces, and
	let us consider the cartesian product $X=\prod_{i\in I} X_i$.
	The \textbf{product topology} on $X$ (i.e., $\T_{X}$) is the topology generated by the subbase
	defined as follows:
	$$
		\S = \setdef{\pi^{-1}_j[U]}{U\in\T_j, j\in I}.
	$$
	I.e., $\T_X$ is generated by the subbase $\S$, with the subbasis elements
	being the inverse projections of open sets of $\T_j$.
\end{definition}

The following theorem is an important result, not only because it gives us the recipe
for the construction of a base for the product topology, but also because it shows that
the basis open sets for the product topology are not the cartesian products of open sets
of the underlying topological spaces.

\begin{theorem}
	\label{th:product-topology-base}
	Let $\{(X_i, \T_i)\}_{i\in I}$ be a family of topological spaces, and let $\{\B_i\}_{i\in I}$ be
	a family such that $\B_i$ is a base for $\T_i$. Then, a base for the product topology
	on $X=\prod_{i\in I} X_i$ is composed by the cartesian products $\prod_{i\in I}B_i$,
	so that $\exists I_0\subset I$ finite such that
	$B_i\in \B_i$ for $i\in I_0$ and $B_i=X_i$ for $i\in I\setminus I_0$.
\end{theorem}

\begin{proof}Let us consider the topology $\T$ on $X$ generated by the following subbase:
	$$
	\S = \setdef{\pi^{-1}_i[B]}{B\in\B_i,\,i\in I}.
	$$
	We can prove that $\T=\T_X$ by using double inclusion.
	\begin{itemize}
		\item[$\boxed{\subseteq}$] By definition, every $S\in \S$ is open in $\T_X$, thus $\B\in \T_X$ $\Then$ $\T\subseteq \T_X$.
		\item[$\boxed{\subseteq}$] Let $\S_X$ be a subbase of $\T_X$. According to \defref{def:product-topology}:
		$$
			\S_X = \setdef{\pi^{-1}_j[A]}{A\in\T_j, j\in I}.
		$$
		Since $A\in \T_i$ and $\B_i$ is base for $\T_i$, $A$ can be expressed as union of
		basis opens: $A=\bigcup_{B\in \B_i} B$. Therefore,
		$\pi^{-1}_i[A] = \pi^{-1}_i\left[\bigcup_{B\in \B_i} B\right] = \bigcup_{B\in \B_i} \pi^{-1}_i[B] = \bigcup_{S\in\S}S$.
		Since $\S$ is subbase for $\T$ ($\S\subset \T$) $\Then$ $\pi^{-1}_i[A]=\bigcup_{S\in\S}S\in \T$ $\Then$ $\T_X\subseteq \T$.
	\end{itemize}
	Let us define the family $\B_X$ as follows:
	$$
		\B_X = \setdef{\prod_{i\in I}B_i}{\left(B_i\in\B_i,\,\forall i\in I_0\right)\,\land\,\left(B_i=X_i,\,\forall i\in I\setminus I_0\right)}.
	$$
	We want to prove that all the elements of $\B_X$ are in the topology $\T$, hence of the product topology $\T_X$.
	Then, if we prove that $\B_X$ is a base for $\T$, we will have proved that $\B_X$ is a base for $\T_X$.

	Let us first prove that $B\in \T_X$, $\forall B\in \B_X$. Let $B=\prod_{i\in I}B_i\in\B_X$, according to \lemref{lem:projection}:
	$B=\bigcap_{i\in I}\pi^{-1}_i[B_i] = \left(\bigcap_{i\in I_0}B_i\times \prod_{j\in I_0\setminus\{i\}}X_j\right)\cap \left(\prod_{i\in I\setminus I_0}X_i\right)=\bigcap_{i\in I_0} \pi^{-1}_i[B_i]$.
	Therefore, $B$ is an intersection of elements of $\S$, hence $B\in \T = \T_X$. Therefore, $\B_X\subset \T_X$.

	To prove that $\B_X$ is a base for $\T$ (and, therefore, for $\T_X$) we are going to prove that $\B_X$ fulfills
	the conditions of \thref{th:1-3}:
	\begin{enumerate}
		\item Let us prove that $\bigcup\B_X = X$:
		\begin{itemize}
			\item[$\boxed{\subseteq}$] Let $x\in \bigcup\B_X$ $\Then$ $x\in \left(\prod_{i\in I_0} B_i\right)\times\left(\prod_{i\in I\setminus I_0}X_i\right)\subset\prod_{i\in I} X_i = X$ $\Then$ $x\in X$ $\Then$ $\bigcup \B_X\subset X$.
			\item[$\boxed{\supseteq}$] Let $C\in \prod_{i\in I}X_i$ $\Then$ $C=\prod_{i\in I}C_i$ with $C_i\subset X_i$.
			Since $\B_i$ is base for $X_i$, $C_i=\bigcup_{B_i\in \B_i} B_i$, $\forall i\in I$.
			Hence, $C=\prod_{i\in I}\left(\bigcup_{B_i\in \B_i} B_i\right) = \bigcup \left(\prod_{i\in I} B_i\right) \subset \bigcup \B_X$ $\Then$ $X\subset \bigcup \B_X$.
		\end{itemize}
		\item Now consider $U,V\in \B_X$, and $x\in U\cap V$. Then $\exists I_0, I_1 \subset I$ such that
		$U=\bigcap_{i\in I_0}\pi^{-1}_i[B_i]$ and $V=\bigcap_{i\in I_1}\pi^{-1}_i[B_i']$.
		We can define $I^*=I_0\cup I_1$, and complete the definitions $B_i=X_i,\,\forall i\in I_1\setminus I_0$ and
		$B_i'=X_i,\,\forall i\in I_0\setminus I_1$.
		Thus,
		\begin{align*}
			U\cap V &= \left(\bigcap_{i\in I^*}\pi^{-1}_i[B_i]\right)\cap\left(\bigcap_{i\in I^*}\pi^{-1}_i[B_i']\right)\\
			&= \bigcap_{i\in I^*}\left(\pi^{-1}_i[B_i]\cap\pi^{-1}_{i}[B_i']\right)\\
			&= \bigcap_{i\in I^*}\pi^{-1}_i[B_i\cap B_i'].
		\end{align*}
		Since $x\in (U\cap V)$ $\Then$ $\pi_i(x)\in\pi_i(U\cap V)$ $\overset{\lemref{lem:projection}}{\Then}$
		$\pi_i(x)\in (B_i\cap B_i'),$ with $(B_i\cap B_i')\in \B_i$ by definition.
		Therefore, $\exists B_i''\in \B_i$ such that $\pi_i(x)\in B_i''\subset (B_i\cap B_i')$.
		This means that $x_i\in \pi^{-1}_i(x) \in \pi^{-1}_i[B_i'']\subset \pi^{-1}_i[B_i\cap B_i']$,
		so:
		$$
		 x\in \bigcap_{i\in I^*}\pi^{-1}_i[B_i'']\subset \bigcap_{i\in I^*}\pi^{-1}_i[B_i\cap B_i'] = U\cap V.
		$$
		With $\bigcap_{i\in I^*}\pi^{-1}_i[B_i'']\in \B_X$ by definition.
		Thus, the conditions of \thref{th:1-3} are fulfilled.
	\end{enumerate}
	With this, we have proved that $\B_X$ is a base for $\T$, and therefore for $\T_X$.
\end{proof}

We can readily note that in the particular case of a finite product of topological spaces,
the \thref{th:product-topology-base} ensures there is a base for the product topology
on $X_1\times\ldots X_n$ that is composed by subsets of the form $B_1\times\ldots\times B_n$,
where $B_i\in \B_i$, and $\B_i$ is a base for $(X_i, \T_i)$, $\forall i\in\{1,\ldots,n\}$.

\begin{remark}
	Although the product of open sets $X_i$ are a base for the product topology in the case of
	finite products, they are not the only open sets in the topology.
	For instance, it can be shown (and this will become trivial to prove in next chapters) that
	a circle without the border is an open set in $\mathbb{R}^2$, but it cannot be expressed as
	the cartesian product of two open intervals in $\mathbb{R}$, since that would be an open
	rectangle. How is this possible? Because the circle without the border can be expressed as the union
	of open rectangles, which are open sets in $\mathbb{R}^2$.
\end{remark}

\subsubsection{Box topology}\label{subsec:box-topology}

Having introduced the product topology, we can now introduce the \emph{box topology},
which is the topology in the cartesian product of topological spaces generated by the
product of open sets in each of the topological spaces.

\begin{definition}
	\label{def:box-topology}
	Let $\{(X_i, \T_i)\}_{i\in I}$ be a family of topological spaces, and let us consider
	the cartesian product $X=\prod_{i\in I} X_i$.
	The \textbf{box topology} (or \emph{box-product topology}) on $X$ (i.e., $\T_{X}$) is
	the set $X=\prod_{i\in I}X_i$ equipped with the topology $\T_X$ which is generated by the base:
	$$
		\B_X = \setdef{\prod_{i\in I}U_i}{U_i\in\T_i, i\in I}.
	$$
	The name of this topology comes from the fact that in $\mathbb{R}^n$ the basis sets
	$\prod_{i\in I}B_i$ look like boxes.
	The box topological space is somtimes denoted as $\square_{i\in I}X_i$.
\end{definition}

Hereinafter, whenever we discuss the product of topological spaces
$\prod_{i\in I}X_i$ we will be implicitly referring to the (Tychonoff) product topology,
and not to the box topology unless otherwise stated.

Indeed, as we have seen so far, the Tychnoff product topology coincides with the box topology
in the case of a finite amount of factors, i.e. $X=X_1\times\ldots\times X_n$ (every
$X_i$ equipped with its own topology $\T_i$). However, when we have an infinite product
of topological spaces, the Tychnoff product topology has more desirable properties than
the box topology. For instance, if all the topological spaces $X_i$ are compact (see Chapter~\ref{ch:compactness}
for further details), then the Tychnoff product topology is always compact, whilst the
box topology is not necessarily compact. This is why, amongst other reasons, the Tychnoff
product topology is the most commonly used topology in the product of topological spaces.

\begin{definition}\label{def:usual-topology}
	Let us consider $\mathbb{R}$ and its \emph{usual} topology, i.e., the topology generated
	by the base $\B=\setdef{(a,b)}{a,b\in\mathbb{R}, a<b}$.
	Then, the \textbf{usual topology} on $\mathbb{R}^n$ is the Tychnoff product topology
	on $\mathbb{R}^n$, i.e. $\mathbb{R}^n$ equipped with the topology generated by the base
	$$
		\B_{\mathbb{R}^n}=\setdef{I_{(a_1,b_1)}\times\ldots\times I_{(a_n,b_n)}}{a_i,b_i\in\mathbb{R}, a_i<b_i,\,i\in\{1,\ldots,n\}}.
	$$
\end{definition}

\subsection{Subspace topology}\label{subsec:subspace-topology}

In this section we introduce the topological properties of a subset $Y$ of a
topological space $X$. In particular, we will be interested in the topology on $Y$
which is induced by the topology on $X$. This topology is called the \emph{subspace topology},
and it is also referred to as the \emph{relative topology} or \emph{induced topology}.

\begin{theorem}
	\label{th:subspace-topology}
	Let $(X, \T)$ be a topological space, and $Y\subset X$. Then, the family
	$\Trel{Y} = \setdef{U\cap Y}{U\in\T}$ is a topology on $Y$.
\end{theorem}
\begin{proof}
Let us confirm that $\T_{Y}$ fulfils the conditions of \defref{def:topology}:
	\begin{enumerate}
		\item $\emptyset\in\Trel{Y}$, since $\emptyset\in\T$, and $\emptyset\in(\emptyset\cap Y)\in \Trel{Y}$. Besides,
		$Y\in \Trel{Y}$, since $X\in\T$ and $Y=(X\cap Y)\in \Trel{Y}$. Therefore, $\emptyset, Y\in \Trel{Y}$.
		\item Let $\F\subset \Trel{Y}$, such that $\F=\setdef{F\cap Y}{F\in \T}$. Let us consider the family of
		open sets associated to $\F$, i.e., $\F^*=\setdef{U\in\T}{(U\cap Y)\in \F}$. Consider
		$A=\bigcup_{F\in\F^*}F=\bigcup \F^*$, which by definition is an open set in $\T$.
		We can check that:
		$$
		A\cap Y = \left(\bigcup\F^*\right)\cap Y=\bigcup_{F\in \F^*}\underbrace{(F\cap Y)}_{\in\T}=\bigcup \F.
		$$
		Therefore, $\bigcup \F = (A\cap Y)$, and we know that $A=\bigcup \F^* \in \T$ $\Then$ $\bigcup \F\in \Trel{Y}$.
		\item Let $U, V\in \Trel{Y}$ $\Then$ $\exists A, B\in \T$ such that $U=(A\cap Y)$ and $V=(B\cap Y)$. Then,
		$(U\cap V) = (A\cap Y)\cap(B\cap Y) = (A\cap B)\cap Y$, with $(A\cap B)\in \T$ $\Then$ $(U\cap V)\in \Trel{Y}$.
	\end{enumerate}
\end{proof}


\begin{definition}
	\label{def:subspace-topology}
	The pair $(Y, \Trel{Y})$ is called a \textbf{subspace} of $(X, \T)$,
	and $\Trel{Y}$ is called the \textbf{subspace} (or \textbf{relative}, or \textbf{induced}) {topology} on $Y$.
\end{definition}

Following the previous results, we can now proceed and introduce the base and subbase
for the subspace topology induced by their counterparts in the original topological space.
These results can be easily proved by using the definitions of base and subbase, and
the definition of subspace topology, as we will see in the following lemma.

\begin{lemma}
	\label{lem:subspace-topology-base}
	Let $(X, \T)$ be a topological space, and $\B$ and $\S$ a base and subbase for $\T$, respectively.
	Then, the following statements hold true:
	\begin{itemize}
		\item $\Rel{\B}{Y}=\setdef{B\cap Y}{B\in\B}$ is a base for $\Trel{Y}$.
		\item $\Rel{\S}{Y}=\setdef{S\cap Y}{S\in\S}$ is a subbase for $\Trel{Y}$.
	\end{itemize}
\end{lemma}

\begin{proof}
	Let us proceed in order:
	\begin{itemize}
	\item We are going to use the definition of base for a topology in order to prove that $\Rel{\B}{Y}$  is base for the relative topology $\Trel{Y}$. By definition, $\forall U\in\Trel{Y}$ is of the form $U = A\cap Y,$ with $A\in\B$. We know $\B$ is base for $\T$, hence $\exists \F\subset \B$ such that $A=\bigcup \F$, so $U=(\bigcup \F)\cap Y$. Consider $\F^*=\setdef{U\cap Y}{U\in\F}\subset\Rel{\B}{Y}$, therefore:
	$$
	  U=\underbrace{\left(\bigcup \F\right)}_{\in\B}\cap Y=\bigcup_{F\in \F}F\cap Y=\bigcup \F^*
	$$
	Therefore, $\forall U\in\Trel{Y}$ we can find $\F^*\subset \Rel{\B}{Y}$ such that $U=\bigcup \F^*$, which means that $\Rel{\B}{Y}$ is base for $\Trel{Y}$.
	\item Let $\B$ be the base induced by $\S$, i.e. $\B=\setdef{\bigcap S}{S\in\S}$. We can check that $\Rel{\B}{Y} = \setdef{B\cap Y}{B\in\B}$ (which we have already proved to be a base for $\Trel{Y}$) is the base induced by $\Rel{\S}{Y} =\setdef{S\cap Y}{S\in\S}$, i.e. $\B_{\S_Y}=\setdef{\bigcap S'}{S'\in\S_Y}$. Let us prove this:
		\begin{align*}
		\B_{\S_Y}=&\setdef{\bigcap S'}{S'\in\S_Y}=\setdef{\bigcap (S\cap Y)}{S\in\S}\\
		=&\setdef{\left(\bigcap S\right)\cap Y}{S\in\S}=\setdef{B\cap Y}{B\in\B}\\
		=&\Rel{\B}{Y}
		\end{align*}
	This means that $\Rel{\S}{Y}$ is subbase for $\Trel{Y}$.
	\end{itemize}
\end{proof}

In what follows, we will always consider the subsets of topological spaces as topological spaces equipped with the subspace topology, unless otherwise stated.
This is coherent, given that if we consider $X$ to be a topological space, and we have $Z\subset Y\subset X$, then the relative topology on $Z$ with respect to $X$ is the same as the relative topology on $Z$ with respect to $Y$ when $Y$ is equipped with the relative topology with respect to $X$.

Regarding the product topology, we have the following result about the coherence of the product of topological spaces with the family of subspaces of each of the factors.

\begin{theorem}
	\label{th:product-subspace-topology}
	Let $\{(X_i, \T_i)\}_{i\in I}$ be a family of topological spaces, with $X=\prod_{i\in I} X_i$,
	$\{Y_i\}_{i\in I}$ be a family with $Y_i\subset X_i$, $\forall i\in I$, and $Y=\prod_{i\in I}Y_i\subset X$.
	Considering each $Y_i$ as a topological space equipped with the subspace topology
	induced by $X_i$, then the product topology on $Y$ is the same as the subspace topology induced by $X$ on $Y$.
\end{theorem}

\begin{proof}
	By definition (see \defref{def:product-topology} and \thref{th:product-topology-base}), the product topology on $Y=\prod_{i\in I}Y_i$ is generated by a base $\B_Y$ such that:
	$$
	\B_Y=\setdef{\prod_{i\in I_0}D_i\times \prod_{i\in(I\setminus I_0)}Y_i}{D_i\in\B_{Y_i},\,I_0\subset I}.
	$$
	where $\B_{Y_i}=\setdef{B_i\cap Y_i}{B_i\in \B_i}$ is the base induced by $\B_i$ on $Y_i$.

	Also by definition (see \defref{def:subspace-topology}), the subspace topology on $Y\subset X$ is generated by a base $\Rel{\B}{Y}$ such that:
	\begin{align*}
	\Rel{\B}{Y}=&\setdef{B\cap Y}{B\in\B_X}\\
	=&\setdef{\prod_{i\in I}B_i\cap \prod_{i\in I}Y_i}{\left(B_i\in\B_{X_i},\forall i\in I_0\subset I\right)\land\left(B_i=X_i, \forall i\in I\setminus I_0\right)}.
	\end{align*}
	with $\B_X$ being the base for the product topology on $X$.

	\noindent{\boxed{\subseteq}} Let us consider $F\in \B_{Y}$, then:
	\begin{align*}
	F=&\prod_{i\in I_0} D_i \times \prod_{i\in(I\setminus I_0)}Y_i=\prod_{i\in I_0} (B_i\cap Y_i) \times \prod_{i\in(I\setminus I_0)}(X_i\cap Y_i)=\prod_{i\in I}(A_i\cap Y_i)
	\end{align*}
	with $A_i$ being $B_i\in \B_{X_i}$ if $i\in I_0$, and $X_i$ if $i\in I\setminus I_0$.
	Thus, we have proved that $F\in\B_Y$ $\Then$ $F\in\Rel{\B}{Y}$, i.e. $\T_Y\subset \Trel{Y}$.

	\noindent{\boxed{\supseteq}} Let us consider $F\in \Rel{\B}{Y}$, then:
	\begin{align*}
		F=&\prod_{i\in I}B_i\cap \prod_{i\in I}Y_i = \left(\prod_{i\in I_0} B_i\times \prod_{i\in I\setminus I_0} X_i\right)\cap \prod_{i\in I}Y_i\\
		=&\left(\prod_{i\in I_0} B_i\times \prod_{i\in I\setminus I_0} X_i\right)\cap \left(\prod_{i\in I_0} Y_i\times \prod_{i\in I\setminus I_0} Y_i\right)\\
		=&\prod_{i\in I_0} (B_i\cap Y_i) \times \prod_{i\in I\setminus I_0} (X_i\cap Y_i) = \prod_{i\in I_0} D_i \times \prod_{i\in I\setminus I_0} Y_i
	\end{align*}
	with $D_i\in \B_{Y_i}$, $I_0\subset I$. Thus, we have proved that $F\in \B_Y$ $\Then$ $F\in \Rel{\B}{Y}$, i.e. $\Trel{Y}\subset \T_Y$.

	This means that $\T_Y=\Trel{Y}$, i.e. the product topology on $Y$ is the same as the subspace topology induced by $X$ on $Y$.
\end{proof}

\subsection{Hausdorff spaces}\label{subsec:hausdorff-spaces}

In his original definition of topological space\footnote{Hausdorff, Felix (1914). \emph{Grundz\"{u}ge der Mengenlehre}, Leipzig: Veit, ISBN 978-0-8284-0061-9. Reprinted by Chelsea Publishing Company in 1944, 1949 and 1965}, Hausdorff introduced the concept in terms of neighbourhoods, requiring a certain property which seemed natural in his time but which is not present in the modern definition.
This is because there exists some topologies which do not satisfy this property, as was discovered later on.
Nevertheless, such a property is useful to distinguish between topological spaces which do satisfy it and those which do not.
A generalisation of such a restriction gave rise to the concept of the \emph{separation axioms}, which are a set of properties that a topological space can have or not.
These axioms are sometimes called \emph{Tychnoff separation axioms}, and denoted with the letter $T$ after the German Trennungsaxiom (``separation axiom''), and a numerical subscript (e.g. $T_0$, $T_1$, $T_2$, etc.) to indicate a stronger and stronger separation property.
Despite their names, the separation axioms are not fundamental axioms as those of set theory, but rather defining properties of topological spaces.
In what follows, we are going to introduce three of such axioms, which are the most commonly used in the literature.
However, a more extensive list of separation axioms can be found elsewhere\footnote{Willard, Stephen (1970). \emph{General topology}. Reading, Mass.: Addison-Wesley Pub. Co. ISBN 0-486-43479-6.} if the reader is interested.

\begin{definition}
	Let $(X, \T)$ be a topological space. We say that the space fulfils the separation axiom (or, simply, is a):
	\begin{itemize}
		\item[$T_0$:] If $x,y\in X$, and $x\neq y$, $\exists U\in\T$\,:\,$(x\in U) \land (y\notin U)$, or $(x\notin U) \land (y\in U)$.
		\item[$T_1$:] If $x,y\in X$, and $x\neq y$, $\exists U\in\T$\,:\,$(x\in U)\land (y\notin U)$.
		\item[$T_2$:] If $x,y\in X$, and $x\neq y$, $\exists U,V\in\T$\,:\,$(x\in U)\land (y\in V) \land (U\cap V=\emptyset)$.
	\end{itemize}
	$T_0$ is called the \emph{Kolmogorov axiom}, $T_1$ is called the \emph{Fr\'{e}chet axiom}, and $T_2$ is called the \emph{Hausdorff axiom}.
	A topological space which fulfils the Hausdorff axiom is called a \textbf{Hausdorff space}, and the same applies to the other axioms, i.e. a $T_0$ space is a \emph{Kolmogorov space}, and a $T_1$ space is a \emph{Fr\'{e}chet space}.
\end{definition}

\begin{remark}
	In plain English, a topological space is a Hausdorff space if every pair of points in the space can be separated by disjoint open sets.
	We can readily check that metric spaces and ordered spaces are Hausdorff spaces.
\end{remark}

\begin{example}
	Let $\mathcal{M}$ be a \emph{metric space}\footnote{The concept of metric space will be introduced in Chapter~\ref{ch:metric-spaces}} equipped with the topology $\T_d$ induced by the open balls $B_\epsilon(x)=\setdef{y\in \mathcal{M}}{(x\in\mathcal{M})\land (d(x,y)<\epsilon)}$, where $d$ is the \emph{metric} of $\mathcal{M}$. Then, $\forall x,y \in \mathcal{M}$ we can always find $B_{\epsilon/2}(x)$ and $B_{\epsilon/2}(y)$ such that $\epsilon=d(x,y) > 0$.
	Therefore, $x\in B_{\epsilon/2}(x)\in\T$ and $y\in B_{\epsilon/2}(y)\in \T_d$, such that $B_{\epsilon/2}(x)\cap B_{\epsilon/2}(y)=\emptyset$, so $\mathcal{M}$ is a Hausdorff space (or simply $T_2$).
\end{example}

It is straightforward to prove that a every $T_2$ space is also $T_1$, and that every $T_1$ space is also $T_0$. We leave this simple proof to the reader.

\begin{remark}
	A topological space will fail the $T_0$ axiom \iff there are two distinct points $x,y\in X$ such that $\forall U\in\T: x\in U$, then $y\in U$. If this is the case, we say that $x,y$  are \emph{topologically indisitinguishable}, or that they are \emph{topologically equivalent}, given that any topological property of $x$ will also be a topological property of $y$. This is why we often find the $T_0$ axiom expressed as: ``$X$ is $T_0$ if any two distinct points in $X$ are topologically distinguishable.''
\end{remark}

\begin{example}
	Let $(X,\T)$ be a topological space, with the trivial topology $\T=\{X,\emptyset\}$.
	Then, $\forall x,y\in X$, with $x\neq y$, we have that the only open set containing $x$ is $X$, and the only open set containing $x$ or $y$ is $X$ itself. Thus, $x$ and $y$ are topologically equivalent, and $(X,\T)$ is not $T_0$.
\end{example}

\begin{example}
	Let $X=\mathbb{R}^2$, with the topology $\T=\setdef{B_\epsilon(x)}{x\in\mathbb{R}^2}$, induced by open balls $B_\epsilon(x)=\setdef{y\in\mathbb{R}^2}{d(x,y)=|x_1-y_1|<\epsilon}$ induced by the pseudo-metric $d$.
	Then, all points in a vertical line on the plane, $\setdef{x\in \mathbb{R}^2}{x_1=c,\,c\in \mathbb{R}}$, are topologically equivalent.
	Each of these lines comprise a topological unit.
\end{example}

\begin{example}
	A simple example of a topological space which is $T_0$ but not $T_1$ is the one defined on $X=\{0,1\}$ with the topology $\T=\{\emptyset, \{0\}, \{1\},\{0,1\}\}$, also called \emph{Sierepi\'{n}ski space}. We can see how the two points are topologically distinguishable, but there is no open set containing $\{0\}$ but not $\{1\}$, or \emph{vice versa}. Thus, the space is $T_0$ but not $T_1$.
\end{example}

Now that we have introduced the most commonly used separation axioms, and we have gained a bit of intuition about them with the previous examples, let us introduce a result which shows that the separation axioms are preserved when we consider subspaces of topological spaces.

\begin{theorem}
	\label{th:1-8}
	Let $(X, \T)$ be a topological space, and $Y\subset X$ a subspace of $X$ with the subspace topology $\Trel{Y}$.
	Then, if $X$ is $T_i$ $\Then$ $Y$ is also $T_i$, $i\in\{0,1,2\}$.
\end{theorem}
\begin{proof}
	Let us prove the case $T_2$.
	Consider $x,y\in Y\subset X$, with $x\neq y$.
	Since $X$ is $T_2$, $\exists U,V\in\T$ such that $(x\in U)\land (y\in V)\land (U\cap V=\emptyset)$.
	Given that $U,V\in\T$ $\Then$ $(U\cap Y)\in \Trel{Y}$ and $(V\cap Y)\in \Trel{Y}$.
	Also, $x\in (U\cap Y)$ and $y\in (V\cap Y)$, and $(U\cap Y)\cap (V\cap Y)=\emptyset$.
	Therefore, $\exists U', V'\in \Trel{Y}$\,:\,$(x\in U')\land (y\in V')\land (U'\cap V'=\emptyset)$ $\Then$ $Y$ is $T_2$.
	The same reasoning can be applied to prove the result for $i=0,1$.
\end{proof}

\begin{theorem}
	\label{th:1-9}
	Let $\{X_i\}_{i\in I}$ be a family of topological spaces, such that \textbf{$X=\prod_{i\in I}X_i\neq \emptyset$}.
	Then, $X$\footnote{With the product topology} is a $T_j$ space $\iff$ $\forall i\in I$, $X_i$ is a $T_j$ space, for $j\in\{0,1,2\}$.
\end{theorem}
\begin{proof}
	As we did with the previous result in \thref{th:1-8}, we are going to prove the case $T_2$ only, since the other cases can be proved in a similar way.
	Given that we have to prove an $\iff$ statement, we are going to prove both implications:
	\begin{itemize}
	\item[$\boxed{\Rightarrow}$] Let $X$ be a $T_2$ space, and let $u_j, v_j \in X_j$ two distinct points, such that they are the $j$-th coordinate of the points $u,v\in X$, respectively, i.e. $\pi_j(u)=u_j$ and $\pi_j(v)=v_j$. Since $X$ is $T_2$, $\exists U, V\in \T_X$ such that $(u\in U)\land (v\in V)\land (U\cap V=\emptyset)$. Therefore,
	$$
	U\cap V= \prod_{i\in I}U_i\cap\prod_{i\in I}V_i=\prod_{i\in I}(U_i\cap V_i)=\emptyset,
	$$
	with $U_i, V_i$ satisfying the conditions of \thref{th:product-topology-base}.
	This means that $(U_i\cap V_i)=\emptyset$, $\forall i\in I$. In particular, $(U_j\cap V_j)=\emptyset$.
	Thus, $\forall u_j, v_j\in X_j$, $\exists U_j, V_j \in \T_{X_j}$ such that $(u_j\in U_j)\land (v_j\in V_j)\land (U_j\cap V_j=\emptyset)$, i.e. $X_j$ is $T_2$.
	\item[$\boxed{\Leftarrow}$] Let $X_i$ be a $T_2$ space $\forall i\in I$. Consider $u,v\in X$ two distinct points, $\exists j\in I$ so that $X_j\ni \pi_j(u) = u_j = \neq v_j=\pi_j(v)\in X_j$. Then, $\exists U,V \in X_j$ such that $(u_j\in U)\land (v_j\in V)\land (U_j\cap V_j)=\emptyset$.
	Using the definition of cartesian projection \defref{def:projection}:
	\begin{align*}
		u_j\in U &\Then u_j\in\pi_j[U]=\setdef{x\in X}{x_j\in U}\\
		v_j\in V &\Then v_j\in\pi_j[V]=\setdef{x\in X}{x_j\in V}
	\end{align*}
	Therefore, $\nexists x\in X$ such that $x\in \pi^{-1}_j[U]$ and $x\in \pi^{-1}_j[V]$,
	because that would require $x_j\in U$ and $x_j\in V$, but $U\cap V=\emptyset$.
	If we call $U'=\pi^{-1}_j[U]$ and $V'=\pi^{-1}_j[V]$, it is clear that $U', V'\in \T_X$ according to the definition of product topology \defref{def:product-topology}, hence $\exists U', V' \in \T_x$: $(u\in U')\land (v\in V')\land (U'\cap V'=\emptyset)$, i.e. $X$ is $T_2$.
	\end{itemize}
	This proves that $X$ is $T_2$ $\iff$ $X_i$ is $T_2$, $\forall i\in I$. The same reasoning can be applied to prove the result for $j=0,1$.
\end{proof}

\section{Topological concepts}\label{sec:topological-concepts}

Thus far, we have been introduced to the concept of topological space, and the formal definition of \emph{surrounding} (or \emph{neighbourhood}) of a point in a topological space without the need of resorting to the concept of distance, besides the concepts of \emph{base} and \emph{subbase}, which are useful to define topologies in a more compact way.
Later, we have taken these concepts, and their related results, to a more general realm as is the case of product spaces.
Finally, we have introduced a set of properties that a topological space can have, called \emph{separation axioms}, which are useful to distinguish between topological spaces.
In what follows, we are going to introduce and prove some more results which are core to the study of topological spaces, and which will be useful in this and the following chapters.

\subsection{Closed sets}\label{subsec:closed-sets}

The concept of \emph{closed set}, which we are about to introduce, plays a very important role in the study
of topological spaces.
As the name suggests, it formalises the concept of being ``closed'' as opposed to being ``open'', which we have already introduced before.

\begin{definition}
	\label{def:closed-set}
	Let $(X,\T)$ a topological space, and $A\subset X$.
	The subset $A$ is said to be \textbf{closed}, i.e. $A$ is a \textbf{closed subset}, in $X$ if $A^\complement=(X\setminus A)\subset X$ is an open subset.
\end{definition}

A straightforward result of this definition is that closed sets in a topological space satisfy \emph{dual properties} to those of open sets.

\begin{lemma}
	\label{lem:toplogy-closed-sets}
	Let $(X, \T)$ be a topological space. Then, the following statements hold true:
	\begin{enumerate}
		\item $X$ and $\emptyset$ are closed sets in $X$.
		\item Let $\{A_i\}_{i\in I}$ be a family of closed sets in $X$, then $\bigcap_{i\in I} A_i$ is also a closed set in $X$.
		\item Let $A, B\subset X$ be two closed sets in $X$, then $(A\cup B)$ is also closed in $X$.
	\end{enumerate}
\end{lemma}

\begin{proof}
	Let us proceed in order:
	\begin{enumerate}
		\item $X$ is closed given that $X^\complement=\emptyset\in \T$. The same applies to $\emptyset$, because $\emptyset^\complement=X\in \T$.
		Therefore, $X$ and $\emptyset$ are closed sets in $X$.
		\item Consider the family $\{A_i\}_{i\in I}$, with $A_i$ closed sets in $X$, i.e. $A_i^\complement\in T$ $\Then$ $\left(\bigcup_{i\in I}A_i^\complement\right)\in \T$, by hypothesis. Consider now $A=\bigcap_{i\in I}A_i$ $\Then$ $A^\complement = \left(\bigcap_{i\in I} A_i \right)^\complement=\bigcup_{i\in I}A_i^\complement$, by using the De Morgan's laws. Therefore, $A^\complement\in \T$ $\Then$ $A$ is a closed set in $X$.
		\item Consider $A,B\subset X$ two closed sets, which means that $A^\complement, B^\complement\in \T$ $\Then$ $(A^\complement\cap B^\complement)\in \T$, by hypothesis. Consider now $D=(A\cup B)$, hence $D^\complement = (A\cup B)^\complement = (A^\complement \cap B^\complement)\in \T$, by using the De Morgan's laws. Therefore, $D^\complement\in \T$ $\Then$ $D$ is a closed set in $X$.
	\end{enumerate}
	With this, it is proved that closed sets in a topological space satisfy dual properties to those of open sets.
\end{proof}

The following result relates closed sets in a topological space $X$, and closed sets in the subspace topology for $Y\subset X$:

\begin{lemma}
	Let $(X, \T)$ be a topological space, and $Y\subset X$. The closed sets of $\Trel{Y}$ are the sets of the form $Y\cap C$, with $C$ closed in $X$.
\end{lemma}
\begin{proof}
	Recall that the relative topology on $Y$ is generated by the base $\Rel{\B}{Y}=\setdef{B\cap Y}{B\in\B}$, with $\B$ being a base for the topology $\T$ on $X$. The closed sets in $Y$ are going to be precisely the complements of such sets, i.e., $F=Y\setminus(Y\cap B)$, with $B\in\B$. Now:
	$$
	F=Y\setminus(Y\cap B)=\setdef{x\in X}{(x\in Y)\land(x\notin B)}=Y\cap B^\complement
	$$
	where $B^\complement$ are closed sets in $X$ $\Then$ $F=Y\cap C$, with $C$ closed in $X$.
\end{proof}