\chapter{Topological spaces}\label{ch:topological-spaces}

In this first chapter, we are going to study a fundamental concept which will pervade
the rest of this course, the concept of \emph{topological space}.
Once this concept is introduced, we will proceed with some basic topological concepts,
besides studying the most important operations which allow for the construction of new
topological spaces.

\section{Topologies, bases and subbases}
\label{sec:topologies,-bases-and-subbases}

\subsection{Topologies}
\label{subsec:topologies}

The modern definition of topological space is based upon the fundamental concept of
open  set (which is axiomatically introduced).
Thus, a topological space is defined by specifying the family of open sets which form it.
The formal definition is as follows.

\begin{definition}
	\label{def:topology}
	Let $X$ be a set, and let $\T$ be a family of subsets of $X$ (i.e., $\T\subset\P X$).
	Then $\T$ is called a \textbf{topology} on $X$ if the following properties are
	fulfilled:
	\begin{enumerate}
		\item $\emptyset, X\in \T$
		\item Let $\{A_i\}_{i\in I}$ be a family of subsets such that $A_i \in \T$,
		then $\bigcup\limits_{i\in I} A_i\in\T$
		\item Let $A, B\in \T$, then $(A\cap B)\in \T$
	\end{enumerate}
\end{definition}

Having the formal definition for \emph{topology}, it readily follows that a
\textbf{topological space} is a tuple $(X, \T)$, with $X$ being a set and $\T$ a
topology on $X$.
The elements of $\T$ are called the \textbf{open subsets} of $X$.

\begin{remark}
	The property (\defref{def:topology}-2) could be rewritten in a much more concise by
	saying that, if $\F\subset\T$, then $\bigcup\F\in\T$.
	Also, the third property can be readily extended to any finite amount of open sets
	by induction, so that if $\{A_k\}_{k=1,\dots,n}$ with $A_k\in\T$, then
	$\bigcap\limits_{k=1}^n A_k\in \T$.
\end{remark}

For practical purposes, we might simply use $X$ instead of $(X, \T)$, keeping in mind
that $\T$ changes with $X$.
The first topological concept we introduce based upon the concept of \emph{open subset} is
the concept of \emph{neighbourhood}.

\begin{definition}
	\label{def:neighbourhood}
	Let $C, U\subset X$, with $(X, \T)$ being a topological space. Then, $U$ is said to
	be a \textbf{neighbourhood} of $C$ if $\exists A\in\T$ such that $C\subset A \subset U$.
	The subset $U$ is said to be a neighbourhood of a point $x\in X$ if $U$ is a
	neighbourhood of $\{x\}$.
	An open neighbourhood of $x$ is simply an open subset of $X$ containing $x$.
\end{definition}

The neighbourhoods of the points of a topological space determine the open sets of the topology:

\begin{theorem}
	Let $(X, \T)$ be a topological space. A subset $U\subset X$ is an open subset
	\emph{if and only if} (iff) $U$ is neighbourhood of all its points.
\end{theorem}

\begin{proof} To prove this theorem will proceed with the demonstration of the left and
right implies.

\noindent{$\boxed{\Rightarrow}$} This is trivial, given that $U\in\T$ exists and
fulfils the conditions to be a neighbourhood of all its points.

\noindent{$\boxed{\Leftarrow}$} Let $U$ be a neighbourhood of all its points.
Then, $\forall x\in U$, $\exists V\in \T\,:\,\{x\}\subset V\subset U$ $\Rightarrow$
$U = \bigcup\F$ with $\F=\{V\in\T\,:\,V\subset U\}\,\Then\,U$ is open, since the union
of open subsets is open according to \defref{def:topology}-2.
\end{proof}

The intuitive idea to develop so far is that a neighbourhood of a point $x$ is a set
$U$ which contains all the points ``around'' $x$, which also means that $U$ is open
\iff\,this set only contains all the points surrounding every of its points.

\begin{example}
	Let $X$ be a set, the \emph{trivial topology} is $\T=\{\emptyset, X\}$, and the
	topology $\T=\P X$ is called the $\emph{discrete topology}$.
\end{example}

The proof that both examples fulfil the conditions required by the definition
\defref{def:topology} is straightforward, and left to the reader if interested.
On the one hand, the trivial topology is the smallest one possible, since it has the
minimal elements required to fulfil the definition \defref{def:topology}.
On the other hand, the discrete topology is the largest one possible, given that all
the possible subsets of $X$ are open.

\begin{definition}
	The elements $x\in X$ will be called \emph{points} (or \emph{singletons}).
	Thus, a point $x\in X$ is said to be \emph{isolated} if $\{x\}$ is an open subset
	of $X$, i.e. $\{x\}\in \T$.
\end{definition}

\begin{remark}
	A trivial result is that a topological space $(X,\T)$ is \emph{discrete} (i.e.,
	$\T=\P X$) \iff\ $\forall x\in X$, $x$ is isolated.
\end{remark}

\subsection{Bases}\label{subsec:bases}

\begin{definition}
	Let $(X, \T)$ be a topological space, and $\B = \{B \subset X\,:\, B \in \T\}\subset\T$
	a family of open sets.
	The family $\B$ is a \textbf{base} of $X$ \iff\, every open subset of $X$ can be
	expressed as the union of elements of $\B$. In other words, $\B\subset\T$ is a base
	of $(X,\T)$ if for every $U\in\T$ and $u\in U$, there is a $B\in \B$ such that
	$u\in B\subset U$.
	\label{def:base}
\end{definition}

The convenience of introducing this concept becomes clear with the following two theorems:

\begin{theorem}
	Let $(X, \T)$ be a topological space, and $\B$ a base of $(X, \T)$.
	The following two properties are satisfied:
	\begin{enumerate}
		\item $\bigcup\limits_{B\in\B} B \equiv \bigcup\B = X$
		\item Let $A, B\in \B$ and $x\in A\cap B$, then there exists $C\in \B$ such
		that $x\in C\subset (A\cap B)$
	\end{enumerate}
\end{theorem}

\begin{proof}
	This proof is straightforward by using \defref{def:topology} and \defref{def:base}:
	\begin{enumerate}
		\item \noindent{$\boxed{\Rightarrow}$}  $\bigcup\B \subset X$ is trivial.

		\noindent{$\boxed{\Leftarrow}$} Given that $X\in \T$ (hence it is an open
		subset) $\Then$ $X$ can be expressed as the union of a family $\F\subset \B$
		$\Then$ $X \subset \bigcup \F\subset \bigcup \B$.

		\item Let $A, B\in \B$ $\Then$ $(A\cap B)\in \T$ $\Then$ according to
		\defref{def:base}, $\exists C\in \B$ such that $x\in C\subset(A\cap B)$.
	\end{enumerate}
	With this, the theorem is proved to hold true.
\end{proof}

Interestingly, the two properties of the previous theorem are sufficient for a set
family to be a base of a topology:

\begin{theorem}
	Let $X$ be a set and $\B\subset \P X$ a family of subsets of $X$ such that:
	\begin{enumerate}
		\item $\bigcup \B = X$
		\item If $A, B\in \B$ and $x\in (A\cap B)$, there exists $C\in \B$ such that
		$x\in C\subset (A\cap B)$
	\end{enumerate}
	Therefore, there exists a unique topology $\T$ on $X$ with $\B$ being a base of it,
	and such topology is the smallest (w.r.t.\footnote{with respect to} the inclusion)
	for which the elements of $\B$ are open.
	\label{th:1-3}
\end{theorem}

\begin{proof}
	To proceed with the proof, we will first proof that there exist a topology, and later
	that this is unique.
	Let us define $\T = \{\bigcup \F\,:\,\F\subset \B\}$, and now proceed with the proof
	that $\T$ is a topology.
	\begin{enumerate}
		\item Then, by using the first point $\T =\bigcup \B= X$ $\Then$ $X\in \T$. And, it
		is trivial that $\emptyset = \bigcup \emptyset \in \T$.

		\item Let $\F\subset \T$, and let $\F_A=\{B\in \B\,:\, B\subset A\}$ for each
		$A\in\F$, so that $A=\bigcup\F_A$. Let $\F^*=\bigcup\limits_{\F_A\in\F}\F_A\subset\B$.
		Then,  $\bigcup \F =\bigcup\F^*\in\T$.

		\item Let $A, B\in \T$, and $\F = \{C\in\B\,:\, C \subset(A\cap B)\}$.
		Let us prove that $A\cap B = \bigcup\F$:
		\begin{itemize}
			\item[\boxed{\supseteq}] This is a trivial proof: $\forall x\in\bigcup\F$ $\Then$
			$x\in\setdef{C\in\B}{C\subset(A\cap B)}\ \Then\ x\in (A\cap B)$ $\Then$
			$\bigcup\F \subset (A\cap B)$.
			\item[\boxed{\subseteq}] Let $x\in (A\cap B)$, according to $(\emph{2.})$,
			$\exists C\in \B$ such that $x\in C\subset (A\cap B)$ $\Then$ $x\in C\subset
			\bigcup\F$ $\Then$ $(A\cap B)\subset \bigcup\F$.
		\end{itemize}
		Finally, $\bigcup \F\in \T$ $\Then$ $(A\cap B) \in \T$.
	\end{enumerate}
	Therefore, $\T$ is a topology on $X$.
	To demonstrate the final part of the theorem, consider $\T'$ a topology on $X$ with
	$\B$ as base. By definition, $\T\subset \T'$. Therefore, $\T$ is unique and minimal.
\end{proof}

Thus, to define a topology on a set $X$ one only needs to determine a family $\B$ with
the properties just described.
In what follows, we introduce another related concept which will be convenient later.
\begin{definition}
	\label{def:base-open-neighbourhoods}
	Let $(X, \T)$ be a topological space, and $x\in X$.
	A \textbf{base of (open) neighbourhoods} of $x$ is a family $\B\subset\T$ of open
	neighbourhoods of $x$ such that, for every $x\in U\in \T$, $\exists B\in\B$ such that
	$x\in B\subset U$.
\end{definition}

It is clear that if $(X,\T)$ is a topological space, and we have a family
$\{\B_x\}_{x\in X}$ so that $\B_x$ is a base of neighbourhoods of $x$, then
$\B=\bigcup\limits_{x\in X}\B_x$ is a base on (X, $\T$).
Reciprocally, let $\B$ be a base on $(X, \T)$, then $\B_x = \setdef{A\in\B}{x\in A}$ is
a base of neighbourhoods of $x$.

\subsection{Subbases}\label{subsec:subbases}

In this section we will introduce a concept similar to \emph{base}, which will also be
useful when defining topologies.

\begin{definition}
\label{def:subbase}
	Let $(X, \T)$ be a topological space. A family $\S\subset \T$ is a \textbf{subbase}
	of $(X, \T)$ \iff\ the family of all finite intersections of elements of $\S$, namely
	$\I(\S)$, is a base of $(X, \T)$\footnote{The family of all finite intersections of
	elements of $\S$ means: $\I(\S) = \bigcap\limits_{S\in\mathcal{F}} S$, for all finite subsets
	$\mathcal{F}$ of $\S$. In particular for all subsets $\mathcal{F}$ of cardinality
	0 (empty set, $\mathcal{F}_0=\emptyset$) or 1 (singletons, $\mathcal{F}_1=\{A\}$).}.
\end{definition}

The difference between subbases and bases is that the family of subsets does not have
to satisfy any specific condition to be a subbase:

\begin{theorem}
	Let $X$ be a set, and $\S\subset \P X$ a family of subsets of $X$.
	There exists a unique topology on $X$ with $\S$ as subbase, and it is smallest
	(w.r.t.\,the inclusion) for which the elements of $\S$ are open.
	\label{th:1-4}
\end{theorem}

\begin{proof}
	To prove this result we will show that:
	$$
	\B\doteq \I(\S)=\left\{\bigcap\limits_{S\in\mathcal{F}} S=\bigcap\mathcal{F}\,:\,\mathcal{F}\subset \S\right\},
	$$
	satisfies the conditions of \thref{th:1-3}, as follows:
	\begin{enumerate}
		\item $\mathcal{F}_{0}=\emptyset$ is a finite subset, and $X=\bigcap \emptyset\in\B$, therefore $X\subset
		\bigcup\B$. Besides, all other finite subsets are $\mathcal{F}_{1}=\{A\}$ with $A\in\S
		\subset\P X$, $\bigcap\mathcal{F}_{1}\subset X$. Therefore $\bigcup \B\subset X$. Thus $\bigcup\B = X$.
		\item Let $A, B\in \B$, and $x\in (A\cap B)$. By definition of $\B$, $A\in\B \Rightarrow A=\bigcap S$,
		and $B\in \B \Rightarrow B =\bigcap T$, for some finite sets $S, T\subset \S$.
		Then, $A\cap B = (\bigcap S)\cap(\bigcap T)=\bigcap(S\cup T)\in \S$.
		It is then clear that $\exists C\in\S$ such that $x\in C\subset (A\cap B)$.
	\end{enumerate}
	Thus, the conditions of the \thref{th:1-3} are fulfilled which means that
	$\B$ is base of a topology in $X$, with $\S$ as a subbase.
	Finally, to prove the uniqueness, let us consider a topology $\T'$ on $X$, 
	such that $\forall S\in \S, S\in \T'$ $\Rightarrow$ $\forall B\in\B, B\in\T'$ 
	$\Rightarrow$ $\T\subset \T'$, which also proves that $\B$ is the smallest.
\end{proof}


\begin{example}
	Consider $\mathbb{R}^2$ endowed with the canonical Euclidean distance:
	$$
	d(x,y) =\sqrt{(x_1-y_1)^2+(x_2-y_2)^2},\ \forall x,y \in \mathbb{R}^2.
	$$ 
	An Euclidean open ball centred at $x$ with radius $\epsilon$ is defined as follows: 
	$$
	B_\epsilon(x) = \{z\in\mathbb{R}^2\,|\,d(x,z)<\epsilon\},
	$$ 
	which certainly contains 
\end{example}

